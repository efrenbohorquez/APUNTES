% ========================================================================
% CAPÍTULO 3: ESTADO DEL ARTE
% ========================================================================

\chapter{Estado del Arte}

\section{Introducción}

Este capítulo presenta una revisión exhaustiva de la literatura relacionada con [el tema específico de tu tesis], identificando los avances más significativos, las brechas existentes y el posicionamiento de la presente investigación en el contexto académico actual.

\section{Metodología de Revisión}

\subsection{Estrategia de Búsqueda}

La revisión de literatura se realizó utilizando las siguientes bases de datos académicas:

\begin{itemize}
    \item IEEE Xplore Digital Library
    \item ACM Digital Library
    \item SpringerLink
    \item ScienceDirect
    \item Google Scholar
    \item DBLP Computer Science Bibliography
\end{itemize}

\subsection{Criterios de Selección}

\textbf{Palabras clave utilizadas:}
\begin{itemize}
    \item ``data analytics'', ``machine learning'', ``[términos específicos de tu dominio]''
    \item ``big data'', ``data mining'', ``predictive analytics''
    \item ``[términos específicos de tu metodología]''
\end{itemize}

\textbf{Criterios de inclusión:}
\begin{itemize}
    \item Artículos publicados entre 2018-2024
    \item Investigaciones en inglés y español
    \item Estudios peer-reviewed
    \item Relevancia directa con el problema de investigación
\end{itemize}

\textbf{Criterios de exclusión:}
\begin{itemize}
    \item Artículos sin validación experimental
    \item Estudios con metodología no claramente definida
    \item Trabajos fuera del dominio de aplicación
\end{itemize}

\section{Clasificación de Enfoques}

\subsection{Enfoques Tradicionales}

\subsubsection{Métodos Estadísticos Clásicos}

Los primeros trabajos en [tu dominio] se basaron en técnicas estadísticas tradicionales. \citet{autor2019} propusieron un enfoque basado en [metodología específica] que logró [resultados específicos]. Sin embargo, estas aproximaciones presentan limitaciones en términos de [limitaciones específicas].

\textbf{Ventajas:}
\begin{itemize}
    \item Interpretabilidad de resultados
    \item Fundamentos teóricos sólidos
    \item Menor complejidad computacional
\end{itemize}

\textbf{Limitaciones:}
\begin{itemize}
    \item Escalabilidad limitada
    \item Dificultad para manejar datos no lineales
    \item Requiere supuestos restrictivos sobre los datos
\end{itemize}

\subsubsection{Técnicas de Minería de Datos}

\citet{autor2020} desarrollaron un framework basado en [técnica específica] para [problema específico]. Su enfoque demostró mejoras del [porcentaje]% en [métrica específica] comparado con métodos previos.

\begin{table}[htbp]
\centering
\caption{Comparación de enfoques tradicionales}
\begin{tabular}{@{}lcccc@{}}
\toprule
\textbf{Método} & \textbf{Precisión} & \textbf{Escalabilidad} & \textbf{Complejidad} & \textbf{Interpretabilidad} \\
\midrule
Regresión Lineal & 75\% & Baja & Baja & Alta \\
Árboles de Decisión & 82\% & Media & Media & Alta \\
SVM & 85\% & Media & Alta & Baja \\
\bottomrule
\end{tabular}
\label{tab:enfoques_tradicionales}
\end{table}

\subsection{Enfoques de Machine Learning}

\subsubsection{Métodos de Ensemble}

Los métodos de ensemble han mostrado resultados prometedores en [tu dominio]. \citet{autor2021} propusieron un enfoque híbrido que combina [técnicas específicas], alcanzando una mejora del [porcentaje]% en [métrica].

\textbf{Random Forest:}
\begin{itemize}
    \item Ventajas: Robustez, manejo de sobreajuste
    \item Limitaciones: Menor interpretabilidad
    \item Aplicaciones: [Aplicaciones específicas en tu dominio]
\end{itemize}

\textbf{Gradient Boosting:}
\begin{itemize}
    \item Ventajas: Alto rendimiento predictivo
    \item Limitaciones: Sensible a ruido, tiempo de entrenamiento
    \item Variantes: XGBoost, LightGBM, CatBoost
\end{itemize}

\subsubsection{Deep Learning}

El aprendizaje profundo ha revolucionado el campo de [tu dominio específico]. Trabajos recientes han demostrado su efectividad en [aplicaciones específicas].

\textbf{Redes Neuronales Convolucionales (CNN):}
\citet{autor2022} aplicaron CNN para [aplicación específica], logrando [resultados específicos]. Su arquitectura incluye [descripción de la arquitectura].

\textbf{Redes Neuronales Recurrentes (RNN/LSTM):}
Para problemas de secuencias temporales, \citet{autor2023} desarrollaron un modelo LSTM que [descripción del modelo y resultados].

\textbf{Transformers:}
Los modelos basados en atención han mostrado resultados excepcionales. \citet{autor2024} adaptaron la arquitectura Transformer para [tu aplicación específica].

\begin{figure}[htbp]
\centering
% \includegraphics[width=0.8\textwidth]{imagenes/timeline_ml.png}
\caption{Evolución temporal de las técnicas de machine learning en [tu dominio]}
\label{fig:timeline_ml}
\end{figure}

\subsection{Enfoques de Big Data}

\subsubsection{Procesamiento Distribuido}

El manejo de grandes volúmenes de datos ha motivado el desarrollo de arquitecturas distribuidas. \citet{autor2021_bigdata} propusieron una solución basada en Spark que permite procesar [volumen específico] de datos en [tiempo específico].

\subsubsection{Streaming Analytics}

Para aplicaciones en tiempo real, \citet{autor2022_streaming} desarrollaron un sistema que procesa [velocidad específica] eventos por segundo utilizando [tecnología específica].

\section{Análisis por Dominio de Aplicación}

\subsection{[Dominio Específico 1]}

En el sector [específico], los principales desafíos incluyen [desafíos específicos]. Las soluciones más exitosas han sido:

\begin{itemize}
    \item \textbf{\citet{autor2020_dominio1}:} Propusieron [descripción] logrando [resultados]
    \item \textbf{\citet{autor2021_dominio1}:} Desarrollaron [descripción] con mejoras de [porcentaje]%
    \item \textbf{\citet{autor2022_dominio1}:} Implementaron [descripción] resultando en [beneficios]
\end{itemize}

\subsection{[Dominio Específico 2]}

Los trabajos en [dominio específico] se han enfocado en [objetivos específicos]:

\begin{table}[htbp]
\centering
\caption{Resumen de trabajos en [Dominio Específico 2]}
\begin{tabular}{@{}p{3cm}p{4cm}p{3cm}p{3cm}@{}}
\toprule
\textbf{Autor/Año} & \textbf{Metodología} & \textbf{Dataset} & \textbf{Resultados} \\
\midrule
Autor et al. (2020) & CNN + LSTM & [Descripción] & Precisión: 89\% \\
Autor et al. (2021) & Ensemble Methods & [Descripción] & F1-Score: 0.92 \\
Autor et al. (2022) & Transformer & [Descripción] & RMSE: 0.15 \\
\bottomrule
\end{tabular}
\label{tab:trabajos_dominio2}
\end{table}

\section{Técnicas y Metodologías Específicas}

\subsection{[Técnica Específica 1]}

\subsubsection{Fundamentos}

La técnica [específica] fue introducida por \citet{autor_original} y se basa en [principios fundamentales]. Su formulación matemática es:

\begin{equation}
f(x) = \sum_{i=1}^{n} w_i \cdot \phi_i(x) + b
\label{eq:tecnica_especifica}
\end{equation}

donde $w_i$ son los pesos, $\phi_i(x)$ son las funciones de base, y $b$ es el sesgo.

\subsubsection{Desarrollos Recientes}

\textbf{Extensiones y Mejoras:}
\begin{itemize}
    \item \citet{autor2021_extension}: Propusieron una variación que [descripción de la mejora]
    \item \citet{autor2022_mejora}: Desarrollaron una versión optimizada que reduce [aspecto mejorado]
\end{itemize}

\textbf{Aplicaciones:}
\begin{itemize}
    \item Clasificación de [tipo de datos]: [Referencia y resultados]
    \item Predicción de [variable objetivo]: [Referencia y resultados]
    \item Optimización de [proceso]: [Referencia y resultados]
\end{itemize}

\subsection{[Técnica Específica 2]}

\subsubsection{Enfoques Híbridos}

La combinación de diferentes técnicas ha mostrado resultados prometedores:

\begin{itemize}
    \item \textbf{[Combinación 1]:} \citet{autor2023_hibrido1} combinaron [técnica A] con [técnica B] para [objetivo específico]
    \item \textbf{[Combinación 2]:} \citet{autor2023_hibrido2} integraron [técnica C] y [técnica D] logrando [resultados]
\end{itemize}

\section{Datasets y Benchmarks}

\subsection{Datasets Públicos Relevantes}

\begin{itemize}
    \item \textbf{[Nombre Dataset 1]:} 
        \begin{itemize}
            \item Fuente: [Organización/URL]
            \item Tamaño: [Número de instancias/características]
            \item Descripción: [Breve descripción]
            \item Trabajos que lo utilizan: \citet{autor1}, \citet{autor2}
        \end{itemize}
    
    \item \textbf{[Nombre Dataset 2]:}
        \begin{itemize}
            \item Fuente: [Organización/URL]
            \item Tamaño: [Número de instancias/características]
            \item Descripción: [Breve descripción]
            \item Trabajos que lo utilizan: \citet{autor3}, \citet{autor4}
        \end{itemize}
\end{itemize}

\subsection{Métricas de Evaluación Estándar}

Los trabajos revisados utilizan principalmente las siguientes métricas:

\begin{itemize}
    \item \textbf{Para Clasificación:} Precisión, Recall, F1-Score, AUC-ROC
    \item \textbf{Para Regresión:} RMSE, MAE, R²
    \item \textbf{Para Clustering:} Silhouette Score, Adjusted Rand Index
    \item \textbf{Métricas Específicas:} [Métricas específicas de tu dominio]
\end{itemize}

\section{Herramientas y Frameworks}

\subsection{Plataformas de Desarrollo}

\begin{table}[htbp]
\centering
\caption{Herramientas utilizadas en trabajos relacionados}
\begin{tabular}{@{}p{3cm}p{4cm}p{6cm}@{}}
\toprule
\textbf{Herramienta} & \textbf{Tipo} & \textbf{Trabajos que la utilizan} \\
\midrule
Python/Scikit-learn & ML Framework & \citet{autor1}, \citet{autor2}, \citet{autor3} \\
TensorFlow/Keras & Deep Learning & \citet{autor4}, \citet{autor5} \\
Apache Spark & Big Data & \citet{autor6}, \citet{autor7} \\
R/Caret & Statistical ML & \citet{autor8}, \citet{autor9} \\
\bottomrule
\end{tabular}
\label{tab:herramientas}
\end{table}

\section{Análisis Crítico y Brechas Identificadas}

\subsection{Fortalezas de los Enfoques Actuales}

\begin{enumerate}
    \item \textbf{Diversidad de Técnicas:} Existe una amplia gama de métodos disponibles
    \item \textbf{Resultados Prometedores:} Muchos trabajos reportan mejoras significativas
    \item \textbf{Validación Empírica:} La mayoría incluye evaluación experimental
    \item \textbf{Reproducibilidad:} Creciente tendencia hacia código abierto
\end{enumerate}

\subsection{Limitaciones y Brechas}

\begin{enumerate}
    \item \textbf{Escalabilidad:} Pocos trabajos abordan datasets de gran escala
    \item \textbf{Interpretabilidad:} Falta de explicabilidad en modelos complejos
    \item \textbf{Generalización:} Muchos estudios se limitan a datasets específicos
    \item \textbf{Comparación Justa:} Inconsistencias en evaluación experimental
    \item \textbf{Implementación Práctica:} Brecha entre investigación y aplicación real
\end{enumerate}

\subsection{Oportunidades de Investigación}

Basado en el análisis de la literatura, se identifican las siguientes oportunidades:

\begin{itemize}
    \item \textbf{[Oportunidad 1]:} [Descripción específica de la brecha u oportunidad]
    \item \textbf{[Oportunidad 2]:} [Descripción específica de la brecha u oportunidad]
    \item \textbf{[Oportunidad 3]:} [Descripción específica de la brecha u oportunidad]
\end{itemize}

\section{Posicionamiento de la Investigación}

\subsection{Diferenciación}

La presente investigación se diferencia de los trabajos existentes en:

\begin{enumerate}
    \item \textbf{Enfoque Metodológico:} [Describe cómo tu enfoque es diferente]
    \item \textbf{Conjunto de Datos:} [Explica las características únicas de tus datos]
    \item \textbf{Métricas de Evaluación:} [Menciona métricas adicionales o específicas]
    \item \textbf{Aplicación Práctica:} [Destaca el valor práctico de tu trabajo]
\end{enumerate}

\subsection{Contribuciones Esperadas}

Con base en las brechas identificadas, esta tesis contribuirá:

\begin{itemize}
    \item \textbf{Contribución Metodológica:} [Descripción específica]
    \item \textbf{Contribución Empírica:} [Descripción específica]
    \item \textbf{Contribución Práctica:} [Descripción específica]
\end{itemize}

\section{Conclusiones del Capítulo}

La revisión de la literatura revela que, aunque existe un cuerpo sustancial de investigación en [tu área específica], persisten brechas importantes en [áreas específicas]. Los trabajos existentes han logrado avances significativos en [aspectos específicos], pero presentan limitaciones en [aspectos específicos].

Esta investigación se posiciona para abordar específicamente [la brecha principal] mediante [tu enfoque propuesto], contribuyendo al estado del arte en [forma específica].

El siguiente capítulo presenta la metodología diseñada para abordar estas limitaciones y aprovechar las oportunidades identificadas.
