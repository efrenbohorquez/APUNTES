\documentclass[12pt,letterpaper,oneside]{book}

% ========================================================================
% CONFIGURACIÓN DE PAQUETES Y DOCUMENTO
% ========================================================================

% Paquetes esenciales
\usepackage[utf8]{inputenc}
\usepackage[spanish,es-tabla]{babel}
\usepackage[T1]{fontenc}
\usepackage{lmodern}

% Configuración de página y márgenes
\usepackage[left=3cm,right=2.5cm,top=2.5cm,bottom=2.5cm]{geometry}
\usepackage{setspace}
\onehalfspacing

% Paquetes para matemáticas y estadística
\usepackage{amsmath,amssymb,amsthm}
\usepackage{mathtools}
\usepackage{algorithm}
\usepackage{algorithmic}

% Paquetes para gráficos y figuras
\usepackage{graphicx}
\usepackage{subfig}
\usepackage{float}
\usepackage{tikz}
\usepackage{pgfplots}
\pgfplotsset{compat=1.17}

% Paquetes para tablas y datos
\usepackage{booktabs}
\usepackage{longtable}
\usepackage{multirow}
\usepackage{array}
\usepackage{tabularx}

% Paquetes para código y listados
\usepackage{listings}
\usepackage{xcolor}

% Configuración de colores para código
\definecolor{codegreen}{rgb}{0,0.6,0}
\definecolor{codegray}{rgb}{0.5,0.5,0.5}
\definecolor{codepurple}{rgb}{0.58,0,0.82}
\definecolor{backcolour}{rgb}{0.95,0.95,0.92}

% Configuración de estilo para código
\lstdefinestyle{mystyle}{
    backgroundcolor=\color{backcolour},   
    commentstyle=\color{codegreen},
    keywordstyle=\color{magenta},
    numberstyle=\tiny\color{codegray},
    stringstyle=\color{codepurple},
    basicstyle=\ttfamily\footnotesize,
    breakatwhitespace=false,         
    breaklines=true,                 
    captionpos=b,                    
    keepspaces=true,                 
    numbers=left,                    
    numbersep=5pt,                  
    showspaces=false,                
    showstringspaces=false,
    showtabs=false,                  
    tabsize=2
}
\lstset{style=mystyle}

% Paquetes para referencias y bibliografía
\usepackage[backend=biber,style=apa,natbib=true]{biblatex}
\addbibresource{bibliografia/referencias.bib}

% Paquetes para hipervínculos
\usepackage[hidelinks]{hyperref}
\usepackage{url}

% Configuración de encabezados y pies de página
\usepackage{fancyhdr}
\pagestyle{fancy}
\fancyhf{}
\fancyhead[L]{\leftmark}
\fancyfoot[C]{\thepage}
\renewcommand{\headrulewidth}{0.4pt}

% Configuración de títulos de capítulos
\usepackage{titlesec}
\titleformat{\chapter}[display]
  {\normalfont\huge\bfseries}{\chaptertitlename\ \thechapter}{20pt}{\Huge}
\titlespacing*{\chapter}{0pt}{0pt}{40pt}

% Configuración para índices
\usepackage{makeidx}
\makeindex

% Configuración de teoremas y definiciones
\theoremstyle{definition}
\newtheorem{definicion}{Definición}[chapter]
\newtheorem{teorema}{Teorema}[chapter]
\newtheorem{proposicion}{Proposición}[chapter]
\newtheorem{lema}{Lema}[chapter]
\newtheorem{ejemplo}{Ejemplo}[chapter]

% ========================================================================
% INFORMACIÓN DE LA TESIS
% ========================================================================

\title{Título de la Tesis de Maestría en Analítica de Datos}
\author{Tu Nombre Completo}
\date{\today}

% ========================================================================
% INICIO DEL DOCUMENTO
% ========================================================================

\begin{document}

% Configuración inicial
\frontmatter
\pagenumbering{roman}

% Portada
% ========================================================================
% PORTADA PRINCIPAL
% ========================================================================

\begin{titlepage}
    \centering
    
    % Logo de la universidad (ajustar según tu institución)
    % \includegraphics[width=0.3\textwidth]{imagenes/logo_universidad.png}\\[1cm]
    
    {\scshape\LARGE Universidad [Nombre de tu Universidad] \par}
    \vspace{0.5cm}
    {\scshape\Large Facultad de [Tu Facultad] \par}
    \vspace{0.5cm}
    {\scshape\large Programa de Maestría en Analítica de Datos \par}
    
    \vspace{2cm}
    
    % Título de la tesis
    {\huge\bfseries [Título de tu Tesis de Maestría en Analítica de Datos] \par}
    
    \vspace{2cm}
    
    % Subtítulo o descripción adicional
    {\Large Tesis presentada para optar al título de \par}
    {\Large\textbf{Magíster en Analítica de Datos} \par}
    
    \vspace{2cm}
    
    % Información del autor
    {\Large
    \textbf{Presentada por:} \\
    [Tu Nombre Completo] \\[0.5cm]
    
    \textbf{Director(a) de Tesis:} \\
    [Nombre del Director] \\
    [Título académico del Director] \\[0.5cm]
    
    % Co-director si aplica
    \textbf{Co-director(a):} \\
    [Nombre del Co-director] \\
    [Título académico del Co-director] \\
    }
    
    \vfill
    
    % Fecha y lugar
    {\large [Ciudad], [País] \\}
    {\large \today \par}
    
\end{titlepage}

% Página en blanco después de la portada
\newpage
\thispagestyle{empty}
\mbox{}
\newpage


% Página de derechos (opcional)
% ========================================================================
% PÁGINA DE DERECHOS Y DECLARACIÓN
% ========================================================================

\chapter*{Declaración de Originalidad}
\thispagestyle{empty}

\vspace{2cm}

Yo, \textbf{[Tu Nombre Completo]}, declaro que esta tesis titulada \textit{``[Título de tu Tesis]''} y el trabajo presentado en ella son de mi autoría.

\vspace{1cm}

Confirmo que:

\begin{itemize}
    \item Este trabajo fue realizado íntegramente mientras cursaba la Maestría en Analítica de Datos en [Nombre de la Universidad].
    
    \item Cuando he consultado el trabajo publicado por otros, siempre está claramente referenciado.
    
    \item Cuando he citado trabajos de otros, la fuente siempre se indica. Con excepción de tales citas, esta tesis es enteramente mi propio trabajo.
    
    \item He reconocido todas las fuentes principales de ayuda.
    
    \item Cuando la tesis se basa en trabajo realizado en conjunto con otros, he indicado claramente qué fue hecho por otros y qué fue mi contribución.
\end{itemize}

\vspace{2cm}

\noindent
Firma: \underline{\hspace{6cm}} \\[0.5cm]
Nombre: [Tu Nombre Completo] \\[0.5cm]
Fecha: \underline{\hspace{4cm}}

\vfill

\begin{center}
\textbf{Copyright \copyright\ \the\year\ [Tu Nombre Completo]}

Todos los derechos reservados. Ninguna parte de esta publicación puede ser reproducida, distribuida o transmitida en cualquier forma o por cualquier medio, incluyendo fotocopiado, grabación u otros métodos electrónicos o mecánicos, sin el permiso previo por escrito del autor, excepto en el caso de citas breves incorporadas en reseñas críticas y ciertos otros usos no comerciales permitidos por la ley de derechos de autor.
\end{center}

\newpage


% Dedicatoria
% ========================================================================
% DEDICATORIA
% ========================================================================

\chapter*{Dedicatoria}
\thispagestyle{empty}

\vspace{4cm}

\begin{flushright}
\textit{A mis padres, \\
por su apoyo incondicional \\
y por creer siempre en mis sueños.}

\vspace{1cm}

\textit{A mi familia, \\
por ser mi fuente de inspiración \\
y motivación constante.}

\vspace{1cm}

\textit{A todos aquellos que creen \\
en el poder transformador \\
de los datos y la analítica.}
\end{flushright}

\vfill

\newpage


% Agradecimientos
% ========================================================================
% AGRADECIMIENTOS
% ========================================================================

\chapter*{Agradecimientos}

Quiero expresar mi más sincero agradecimiento a todas las personas e instituciones que hicieron posible la realización de esta tesis de maestría.

En primer lugar, agradezco profundamente a mi director(a) de tesis, \textbf{[Nombre del Director]}, por su invaluable orientación, paciencia y dedicación a lo largo de todo este proceso. Sus conocimientos expertos en analítica de datos y su constante apoyo fueron fundamentales para el desarrollo de esta investigación.

A los profesores del Programa de Maestría en Analítica de Datos de [Nombre de la Universidad], especialmente a [Nombres de profesores relevantes], por compartir sus conocimientos y experiencias que enriquecieron significativamente mi formación académica y profesional.

Al comité evaluador de esta tesis, por el tiempo dedicado a la revisión de este trabajo y por sus valiosas observaciones y sugerencias que contribuyeron a mejorar la calidad del mismo.

A [Nombre de la organización/empresa] por permitirme acceder a los datos utilizados en esta investigación y por brindar el apoyo necesario para llevar a cabo los experimentos y análisis presentados.

A mis compañeros de maestría, con quienes compartí innumerables horas de estudio, discusiones académicas y momentos de apoyo mutuo. Su colaboración y amistad hicieron más enriquecedor este camino.

A mi familia, por su comprensión, paciencia y apoyo incondicional durante estos años de estudio. Su aliento constante fue fundamental para mantener la motivación y perseverancia necesarias.

A [Otras personas o instituciones relevantes], por [razones específicas de agradecimiento].

Finalmente, agradezco a [Fuente de financiamiento, si aplica] por el apoyo financiero que hizo posible la realización de esta investigación.

\vspace{1cm}

\begin{flushright}
[Tu Nombre Completo] \\
\today
\end{flushright}

\newpage


% Resumen en español
% ========================================================================
% RESUMEN EN ESPAÑOL
% ========================================================================

\chapter*{Resumen}

\textbf{Palabras clave:} analítica de datos, machine learning, big data, inteligencia artificial, minería de datos, ciencia de datos

\vspace{1cm}

[Aquí debes escribir un resumen de 300-500 palabras de tu tesis. El resumen debe incluir:]

\textbf{Contexto y Problema:} Describe brevemente el contexto de tu investigación y el problema específico que abordas en el campo de la analítica de datos.

\textbf{Objetivos:} Menciona claramente los objetivos principales de tu tesis, tanto el objetivo general como los específicos.

\textbf{Metodología:} Explica de manera concisa la metodología utilizada, incluyendo las técnicas de analítica de datos, algoritmos de machine learning, herramientas de procesamiento de datos, etc.

\textbf{Resultados Principales:} Presenta los hallazgos más importantes de tu investigación, incluyendo métricas de rendimiento, insights descubiertos, o mejoras logradas.

\textbf{Contribuciones:} Destaca las contribuciones originales de tu trabajo al campo de la analítica de datos.

\textbf{Conclusiones:} Resume las conclusiones principales y el impacto potencial de tu trabajo.

\textbf{Aplicaciones:} Menciona las posibles aplicaciones prácticas de tus resultados en la industria o en futuras investigaciones.

Ejemplo de estructura:

Esta tesis presenta [descripción del problema/oportunidad] en el contexto de [dominio específico]. El objetivo principal fue [objetivo general], mediante [metodología principal]. Se utilizaron técnicas de [técnicas específicas] aplicadas a [descripción de los datos]. Los resultados demuestran [resultados principales], logrando [métricas específicas]. Las principales contribuciones incluyen [contribuciones específicas]. Estos hallazgos tienen implicaciones importantes para [aplicaciones/impacto]. El trabajo abre nuevas oportunidades de investigación en [áreas futuras].

\newpage


% Abstract en inglés
% ========================================================================
% ABSTRACT EN INGLÉS
% ========================================================================

\chapter*{Abstract}

\textbf{Keywords:} data analytics, machine learning, big data, artificial intelligence, data mining, data science

\vspace{1cm}

[Here you should write a 300-500 word abstract of your thesis in English. The abstract should include:]

\textbf{Context and Problem:} Briefly describe the context of your research and the specific problem you address in the field of data analytics.

\textbf{Objectives:} Clearly state the main objectives of your thesis, both general and specific objectives.

\textbf{Methodology:} Concisely explain the methodology used, including data analytics techniques, machine learning algorithms, data processing tools, etc.

\textbf{Main Results:} Present the most important findings of your research, including performance metrics, discovered insights, or achieved improvements.

\textbf{Contributions:} Highlight the original contributions of your work to the field of data analytics.

\textbf{Conclusions:} Summarize the main conclusions and potential impact of your work.

\textbf{Applications:} Mention possible practical applications of your results in industry or future research.

Example structure:

This thesis presents [description of problem/opportunity] in the context of [specific domain]. The main objective was [general objective], through [main methodology]. Techniques of [specific techniques] were applied to [data description]. Results demonstrate [main results], achieving [specific metrics]. Main contributions include [specific contributions]. These findings have important implications for [applications/impact]. The work opens new research opportunities in [future areas].

\newpage


% Índice general
\tableofcontents
\newpage

% Índice de figuras
\listoffigures
\newpage

% Índice de tablas
\listoftables
\newpage

% Lista de algoritmos (si aplica)
\listofalgorithms
\newpage

% ========================================================================
% CONTENIDO PRINCIPAL
% ========================================================================

\mainmatter
\pagenumbering{arabic}

% Capítulo 1: Introducción
% ========================================================================
% CAPÍTULO 1: INTRODUCCIÓN
% ========================================================================

\chapter{Introducción}

\section{Contexto y Motivación}

En la era digital actual, la generación de datos ha alcanzado volúmenes sin precedentes. Organizaciones de todos los sectores enfrentan el desafío de extraer valor significativo de sus vastos repositorios de información. La analítica de datos se ha convertido en una disciplina fundamental para transformar datos en insights accionables que impulsen la toma de decisiones estratégicas.

[Desarrolla aquí el contexto específico de tu investigación, incluyendo:]
\begin{itemize}
    \item El sector o dominio de aplicación
    \item Los desafíos específicos que motivaron tu investigación
    \item La relevancia actual del problema
    \item Las oportunidades identificadas
\end{itemize}

\section{Planteamiento del Problema}

\subsection{Definición del Problema}

[Describe claramente el problema específico que aborda tu tesis. Por ejemplo:]

En el contexto de [sector/dominio específico], existe la necesidad de [descripción del problema]. Los métodos tradicionales de [análisis/procesamiento] presentan limitaciones en términos de [aspectos específicos como escalabilidad, precisión, eficiencia, etc.].

\subsection{Preguntas de Investigación}

Las preguntas que guían esta investigación son:

\begin{enumerate}
    \item ¿Cómo se puede [pregunta principal relacionada con tu objetivo]?
    \item ¿Qué técnicas de analítica de datos son más efectivas para [contexto específico]?
    \item ¿Cuál es el impacto de [factor específico] en [resultado esperado]?
    \item ¿Cómo se comparan los resultados obtenidos con [métodos baseline o estado del arte]?
\end{enumerate}

\section{Objetivos}

\subsection{Objetivo General}

[Enuncia tu objetivo general. Ejemplo:]

Desarrollar un modelo/framework/metodología de analítica de datos para [propósito específico] que permita [resultado esperado] en el contexto de [dominio de aplicación].

\subsection{Objetivos Específicos}

\begin{enumerate}
    \item Analizar y caracterizar [aspecto específico del problema o los datos].
    \item Diseñar e implementar [solución/modelo/algoritmo específico].
    \item Evaluar el rendimiento de [tu propuesta] utilizando métricas de [tipo de métricas].
    \item Comparar los resultados obtenidos con [métodos existentes/baselines].
    \item Validar la aplicabilidad de la solución propuesta en [contexto real/casos de uso].
\end{enumerate}

\section{Hipótesis}

[Formula tu hipótesis principal. Ejemplo:]

La aplicación de [técnicas/metodologías específicas] de analítica de datos permitirá [resultado esperado] con una mejora del [porcentaje o métrica específica] en comparación con los métodos tradicionales utilizados en [contexto específico].

\subsection{Hipótesis Secundarias}

\begin{itemize}
    \item [Hipótesis específica 1]
    \item [Hipótesis específica 2]
    \item [Hipótesis específica 3]
\end{itemize}

\section{Justificación}

\subsection{Relevancia Académica}

Esta investigación contribuye al campo de la analítica de datos mediante:

\begin{itemize}
    \item [Contribución teórica específica]
    \item [Nueva metodología o enfoque]
    \item [Extensión de conocimiento existente]
    \item [Validación empírica de teorías]
\end{itemize}

\subsection{Relevancia Práctica}

Los resultados de esta investigación tienen aplicaciones directas en:

\begin{itemize}
    \item [Sector/industria específica]
    \item [Tipo de organizaciones]
    \item [Problemas empresariales concretos]
    \item [Mejoras en procesos existentes]
\end{itemize}

\subsection{Relevancia Social}

El impacto social de esta investigación incluye:

\begin{itemize}
    \item [Beneficios para la sociedad]
    \item [Mejoras en servicios públicos/privados]
    \item [Contribución al bienestar general]
\end{itemize}

\section{Alcance y Limitaciones}

\subsection{Alcance}

Esta investigación abarca:

\begin{itemize}
    \item [Definir claramente qué incluye tu estudio]
    \item [Población o conjunto de datos considerado]
    \item [Periodo temporal del estudio]
    \item [Técnicas y metodologías utilizadas]
    \item [Métricas de evaluación consideradas]
\end{itemize}

\subsection{Limitaciones}

Las limitaciones identificadas incluyen:

\begin{itemize}
    \item [Limitaciones metodológicas]
    \item [Restricciones de datos]
    \item [Limitaciones temporales]
    \item [Restricciones tecnológicas]
    \item [Limitaciones de generalización]
\end{itemize}

\section{Contribuciones Esperadas}

Las principales contribuciones de esta tesis son:

\begin{enumerate}
    \item \textbf{Contribución Metodológica:} [Descripción de nuevos métodos o mejoras]
    \item \textbf{Contribución Técnica:} [Implementaciones, algoritmos, herramientas]
    \item \textbf{Contribución Empírica:} [Resultados experimentales, validaciones]
    \item \textbf{Contribución Aplicada:} [Soluciones prácticas, casos de uso]
\end{enumerate}

\section{Estructura de la Tesis}

Este documento está organizado de la siguiente manera:

\begin{itemize}
    \item \textbf{Capítulo 2 - Marco Teórico:} Presenta los fundamentos teóricos y conceptuales que sustentan la investigación.
    
    \item \textbf{Capítulo 3 - Estado del Arte:} Revisa trabajos relacionados y posiciona la investigación en el contexto actual.
    
    \item \textbf{Capítulo 4 - Metodología:} Describe la metodología de investigación, técnicas utilizadas y diseño experimental.
    
    \item \textbf{Capítulo 5 - Desarrollo:} Detalla la implementación de la solución propuesta.
    
    \item \textbf{Capítulo 6 - Resultados y Análisis:} Presenta los resultados obtenidos y su análisis.
    
    \item \textbf{Capítulo 7 - Conclusiones:} Resume los hallazgos principales, contribuciones y trabajo futuro.
\end{itemize}


% Capítulo 2: Marco Teórico
% ========================================================================
% CAPÍTULO 2: MARCO TEÓRICO
% ========================================================================

\chapter{Marco Teórico}

\section{Introducción}

Este capítulo presenta los fundamentos teóricos y conceptuales que sustentan la investigación. Se abordan los conceptos clave de analítica de datos, las técnicas y metodologías relevantes, así como los marcos de referencia utilizados en el desarrollo de la tesis.

\section{Analítica de Datos: Conceptos Fundamentales}

\subsection{Definición y Evolución}

La analítica de datos se define como el proceso sistemático de examinar conjuntos de datos para extraer conclusiones útiles, identificar patrones y apoyar la toma de decisiones \citep{referencia_ejemplo}. Este campo ha evolucionado significativamente desde las técnicas estadísticas tradicionales hasta los métodos avanzados de inteligencia artificial.

\begin{definicion}
\textbf{Analítica de Datos:} Disciplina que combina estadística, informática y conocimiento del dominio para extraer insights significativos de los datos y facilitar la toma de decisiones basada en evidencia.
\end{definicion}

\subsection{Tipos de Analítica}

La analítica de datos se clasifica tradicionalmente en cuatro categorías:

\begin{enumerate}
    \item \textbf{Analítica Descriptiva:} Responde a la pregunta ``¿Qué pasó?''
        \begin{itemize}
            \item Técnicas: Estadística descriptiva, visualización, reportes
            \item Objetivo: Resumir y describir características de los datos
        \end{itemize}
    
    \item \textbf{Analítica Diagnóstica:} Responde a la pregunta ``¿Por qué pasó?''
        \begin{itemize}
            \item Técnicas: Análisis de correlación, regresión, análisis de causas
            \item Objetivo: Identificar factores que contribuyen a los resultados
        \end{itemize}
    
    \item \textbf{Analítica Predictiva:} Responde a la pregunta ``¿Qué va a pasar?''
        \begin{itemize}
            \item Técnicas: Machine learning, modelos estadísticos, series temporales
            \item Objetivo: Predecir eventos o comportamientos futuros
        \end{itemize}
    
    \item \textbf{Analítica Prescriptiva:} Responde a la pregunta ``¿Qué se debe hacer?''
        \begin{itemize}
            \item Técnicas: Optimización, simulación, algoritmos de decisión
            \item Objetivo: Recomendar acciones para lograr resultados deseados
        \end{itemize}
\end{enumerate}

\section{Fundamentos de Machine Learning}

\subsection{Paradigmas de Aprendizaje}

\subsubsection{Aprendizaje Supervisado}

El aprendizaje supervisado utiliza datos etiquetados para entrenar modelos que puedan predecir outcomes para nuevas observaciones.

\begin{definicion}
\textbf{Aprendizaje Supervisado:} Paradigma de machine learning donde el algoritmo aprende de ejemplos de entrada-salida para hacer predicciones sobre nuevos datos.
\end{definicion}

Técnicas principales:
\begin{itemize}
    \item \textbf{Clasificación:} Support Vector Machines, Random Forest, Redes Neuronales
    \item \textbf{Regresión:} Regresión Lineal, Regresión Polinomial, Regresión Logística
\end{itemize}

\subsubsection{Aprendizaje No Supervisado}

Identifica patrones ocultos en datos sin etiquetas previas.

Técnicas principales:
\begin{itemize}
    \item \textbf{Clustering:} K-means, DBSCAN, Clustering Jerárquico
    \item \textbf{Reducción de Dimensionalidad:} PCA, t-SNE, UMAP
    \item \textbf{Detección de Anomalías:} Isolation Forest, One-Class SVM
\end{itemize}

\subsubsection{Aprendizaje por Refuerzo}

El agente aprende a través de interacciones con un entorno, recibiendo recompensas o penalizaciones.

\section{Procesamiento de Big Data}

\subsection{Características del Big Data}

El Big Data se caracteriza por las ``5 V's'':

\begin{itemize}
    \item \textbf{Volumen:} Grandes cantidades de datos
    \item \textbf{Velocidad:} Generación rápida de datos
    \item \textbf{Variedad:} Diferentes tipos y formatos de datos
    \item \textbf{Veracidad:} Calidad y confiabilidad de los datos
    \item \textbf{Valor:} Capacidad de generar insights útiles
\end{itemize}

\subsection{Arquitecturas de Procesamiento}

\subsubsection{Procesamiento por Lotes (Batch)}

\begin{itemize}
    \item \textbf{Características:} Procesamiento de grandes volúmenes en intervalos
    \item \textbf{Tecnologías:} Apache Hadoop, Apache Spark
    \item \textbf{Casos de uso:} ETL, análisis histórico, reportes periódicos
\end{itemize}

\subsubsection{Procesamiento en Tiempo Real (Streaming)}

\begin{itemize}
    \item \textbf{Características:} Procesamiento continuo de datos en movimiento
    \item \textbf{Tecnologías:} Apache Kafka, Apache Storm, Apache Flink
    \item \textbf{Casos de uso:} Monitoreo en tiempo real, detección de fraudes
\end{itemize}

\section{Metodologías de Ciencia de Datos}

\subsection{CRISP-DM (Cross-Industry Standard Process for Data Mining)}

CRISP-DM es una metodología estándar que define un proceso estructurado para proyectos de minería de datos:

\begin{enumerate}
    \item \textbf{Entendimiento del Negocio}
        \begin{itemize}
            \item Definición de objetivos empresariales
            \item Evaluación de la situación
            \item Definición de objetivos de minería de datos
        \end{itemize}
    
    \item \textbf{Entendimiento de los Datos}
        \begin{itemize}
            \item Recolección inicial de datos
            \item Descripción de los datos
            \item Exploración de los datos
            \item Verificación de calidad
        \end{itemize}
    
    \item \textbf{Preparación de los Datos}
        \begin{itemize}
            \item Selección de datos
            \item Limpieza de datos
            \item Construcción de datos
            \item Integración de datos
        \end{itemize}
    
    \item \textbf{Modelado}
        \begin{itemize}
            \item Selección de técnicas de modelado
            \item Generación del diseño de pruebas
            \item Construcción del modelo
            \item Evaluación del modelo
        \end{itemize}
    
    \item \textbf{Evaluación}
        \begin{itemize}
            \item Evaluación de resultados
            \item Revisión del proceso
            \item Determinación de próximos pasos
        \end{itemize}
    
    \item \textbf{Despliegue}
        \begin{itemize}
            \item Planificación del despliegue
            \item Planificación de monitoreo y mantenimiento
            \item Producción del informe final
        \end{itemize}
\end{enumerate}

\subsection{KDD (Knowledge Discovery in Databases)}

El proceso KDD incluye las siguientes etapas:

\begin{enumerate}
    \item Selección de datos
    \item Preprocesamiento
    \item Transformación
    \item Minería de datos
    \item Interpretación y evaluación
\end{enumerate}

\section{Técnicas Específicas Relevantes}

\subsection{[Incluir técnicas específicas para tu investigación]}

Por ejemplo, si tu tesis se enfoca en:

\subsubsection{Procesamiento de Lenguaje Natural (NLP)}
\begin{itemize}
    \item Tokenización y preprocesamiento de texto
    \item Modelos de representación: TF-IDF, Word2Vec, BERT
    \item Análisis de sentimientos
    \item Extracción de entidades
\end{itemize}

\subsubsection{Computer Vision}
\begin{itemize}
    \item Procesamiento de imágenes
    \item Redes neuronales convolucionales (CNN)
    \item Detección y clasificación de objetos
    \item Segmentación de imágenes
\end{itemize}

\subsubsection{Series Temporales}
\begin{itemize}
    \item Análisis de tendencias y estacionalidad
    \item Modelos ARIMA
    \item Redes neuronales recurrentes (RNN, LSTM)
    \item Forecasting multivariado
\end{itemize}

\section{Métricas de Evaluación}

\subsection{Métricas para Clasificación}

\begin{itemize}
    \item \textbf{Exactitud (Accuracy):} $\frac{TP + TN}{TP + TN + FP + FN}$
    \item \textbf{Precisión (Precision):} $\frac{TP}{TP + FP}$
    \item \textbf{Recall (Sensibilidad):} $\frac{TP}{TP + FN}$
    \item \textbf{F1-Score:} $\frac{2 \times Precision \times Recall}{Precision + Recall}$
    \item \textbf{AUC-ROC:} Área bajo la curva ROC
\end{itemize}

\subsection{Métricas para Regresión}

\begin{itemize}
    \item \textbf{Error Cuadrático Medio (MSE):} $\frac{1}{n}\sum_{i=1}^{n}(y_i - \hat{y_i})^2$
    \item \textbf{Raíz del Error Cuadrático Medio (RMSE):} $\sqrt{MSE}$
    \item \textbf{Error Absoluto Medio (MAE):} $\frac{1}{n}\sum_{i=1}^{n}|y_i - \hat{y_i}|$
    \item \textbf{Coeficiente de Determinación (R²):} $1 - \frac{SS_{res}}{SS_{tot}}$
\end{itemize}

\section{Herramientas y Tecnologías}

\subsection{Lenguajes de Programación}

\begin{itemize}
    \item \textbf{Python:} Pandas, NumPy, Scikit-learn, TensorFlow, PyTorch
    \item \textbf{R:} Tidyverse, Caret, RandomForest
    \item \textbf{SQL:} Manejo de bases de datos relacionales
    \item \textbf{Scala:} Para procesamiento distribuido con Spark
\end{itemize}

\subsection{Plataformas de Big Data}

\begin{itemize}
    \item \textbf{Apache Hadoop:} Almacenamiento y procesamiento distribuido
    \item \textbf{Apache Spark:} Motor de análisis unificado
    \item \textbf{Apache Kafka:} Streaming de datos en tiempo real
    \item \textbf{Elasticsearch:} Motor de búsqueda y análisis
\end{itemize}

\subsection{Herramientas de Visualización}

\begin{itemize}
    \item \textbf{Tableau:} Visualización empresarial
    \item \textbf{Power BI:} Plataforma de Microsoft
    \item \textbf{D3.js:} Visualizaciones web interactivas
    \item \textbf{Matplotlib/Seaborn:} Visualización en Python
\end{itemize}

\section{Consideraciones Éticas y de Privacidad}

\subsection{Ética en Analítica de Datos}

\begin{itemize}
    \item Transparencia en algoritmos
    \item Sesgo algorítmico y fairness
    \item Consentimiento informado
    \item Responsabilidad en decisiones automatizadas
\end{itemize}

\subsection{Privacidad y Protección de Datos}

\begin{itemize}
    \item Principios de GDPR
    \item Anonimización y pseudonimización
    \item Privacidad diferencial
    \item Seguridad de datos
\end{itemize}

\section{Conclusiones del Capítulo}

Este capítulo ha establecido las bases teóricas necesarias para comprender [los conceptos específicos relevantes a tu investigación]. Los fundamentos presentados proporcionan el marco conceptual para el desarrollo de [tu propuesta específica] que se detallará en los capítulos siguientes.


% Capítulo 3: Estado del Arte
% ========================================================================
% CAPÍTULO 3: ESTADO DEL ARTE
% ========================================================================

\chapter{Estado del Arte}

\section{Introducción}

Este capítulo presenta una revisión exhaustiva de la literatura relacionada con [el tema específico de tu tesis], identificando los avances más significativos, las brechas existentes y el posicionamiento de la presente investigación en el contexto académico actual.

\section{Metodología de Revisión}

\subsection{Estrategia de Búsqueda}

La revisión de literatura se realizó utilizando las siguientes bases de datos académicas:

\begin{itemize}
    \item IEEE Xplore Digital Library
    \item ACM Digital Library
    \item SpringerLink
    \item ScienceDirect
    \item Google Scholar
    \item DBLP Computer Science Bibliography
\end{itemize}

\subsection{Criterios de Selección}

\textbf{Palabras clave utilizadas:}
\begin{itemize}
    \item ``data analytics'', ``machine learning'', ``[términos específicos de tu dominio]''
    \item ``big data'', ``data mining'', ``predictive analytics''
    \item ``[términos específicos de tu metodología]''
\end{itemize}

\textbf{Criterios de inclusión:}
\begin{itemize}
    \item Artículos publicados entre 2018-2024
    \item Investigaciones en inglés y español
    \item Estudios peer-reviewed
    \item Relevancia directa con el problema de investigación
\end{itemize}

\textbf{Criterios de exclusión:}
\begin{itemize}
    \item Artículos sin validación experimental
    \item Estudios con metodología no claramente definida
    \item Trabajos fuera del dominio de aplicación
\end{itemize}

\section{Clasificación de Enfoques}

\subsection{Enfoques Tradicionales}

\subsubsection{Métodos Estadísticos Clásicos}

Los primeros trabajos en [tu dominio] se basaron en técnicas estadísticas tradicionales. \citet{autor2019} propusieron un enfoque basado en [metodología específica] que logró [resultados específicos]. Sin embargo, estas aproximaciones presentan limitaciones en términos de [limitaciones específicas].

\textbf{Ventajas:}
\begin{itemize}
    \item Interpretabilidad de resultados
    \item Fundamentos teóricos sólidos
    \item Menor complejidad computacional
\end{itemize}

\textbf{Limitaciones:}
\begin{itemize}
    \item Escalabilidad limitada
    \item Dificultad para manejar datos no lineales
    \item Requiere supuestos restrictivos sobre los datos
\end{itemize}

\subsubsection{Técnicas de Minería de Datos}

\citet{autor2020} desarrollaron un framework basado en [técnica específica] para [problema específico]. Su enfoque demostró mejoras del [porcentaje]% en [métrica específica] comparado con métodos previos.

\begin{table}[htbp]
\centering
\caption{Comparación de enfoques tradicionales}
\begin{tabular}{@{}lcccc@{}}
\toprule
\textbf{Método} & \textbf{Precisión} & \textbf{Escalabilidad} & \textbf{Complejidad} & \textbf{Interpretabilidad} \\
\midrule
Regresión Lineal & 75\% & Baja & Baja & Alta \\
Árboles de Decisión & 82\% & Media & Media & Alta \\
SVM & 85\% & Media & Alta & Baja \\
\bottomrule
\end{tabular}
\label{tab:enfoques_tradicionales}
\end{table}

\subsection{Enfoques de Machine Learning}

\subsubsection{Métodos de Ensemble}

Los métodos de ensemble han mostrado resultados prometedores en [tu dominio]. \citet{autor2021} propusieron un enfoque híbrido que combina [técnicas específicas], alcanzando una mejora del [porcentaje]% en [métrica].

\textbf{Random Forest:}
\begin{itemize}
    \item Ventajas: Robustez, manejo de sobreajuste
    \item Limitaciones: Menor interpretabilidad
    \item Aplicaciones: [Aplicaciones específicas en tu dominio]
\end{itemize}

\textbf{Gradient Boosting:}
\begin{itemize}
    \item Ventajas: Alto rendimiento predictivo
    \item Limitaciones: Sensible a ruido, tiempo de entrenamiento
    \item Variantes: XGBoost, LightGBM, CatBoost
\end{itemize}

\subsubsection{Deep Learning}

El aprendizaje profundo ha revolucionado el campo de [tu dominio específico]. Trabajos recientes han demostrado su efectividad en [aplicaciones específicas].

\textbf{Redes Neuronales Convolucionales (CNN):}
\citet{autor2022} aplicaron CNN para [aplicación específica], logrando [resultados específicos]. Su arquitectura incluye [descripción de la arquitectura].

\textbf{Redes Neuronales Recurrentes (RNN/LSTM):}
Para problemas de secuencias temporales, \citet{autor2023} desarrollaron un modelo LSTM que [descripción del modelo y resultados].

\textbf{Transformers:}
Los modelos basados en atención han mostrado resultados excepcionales. \citet{autor2024} adaptaron la arquitectura Transformer para [tu aplicación específica].

\begin{figure}[htbp]
\centering
% \includegraphics[width=0.8\textwidth]{imagenes/timeline_ml.png}
\caption{Evolución temporal de las técnicas de machine learning en [tu dominio]}
\label{fig:timeline_ml}
\end{figure}

\subsection{Enfoques de Big Data}

\subsubsection{Procesamiento Distribuido}

El manejo de grandes volúmenes de datos ha motivado el desarrollo de arquitecturas distribuidas. \citet{autor2021_bigdata} propusieron una solución basada en Spark que permite procesar [volumen específico] de datos en [tiempo específico].

\subsubsection{Streaming Analytics}

Para aplicaciones en tiempo real, \citet{autor2022_streaming} desarrollaron un sistema que procesa [velocidad específica] eventos por segundo utilizando [tecnología específica].

\section{Análisis por Dominio de Aplicación}

\subsection{[Dominio Específico 1]}

En el sector [específico], los principales desafíos incluyen [desafíos específicos]. Las soluciones más exitosas han sido:

\begin{itemize}
    \item \textbf{\citet{autor2020_dominio1}:} Propusieron [descripción] logrando [resultados]
    \item \textbf{\citet{autor2021_dominio1}:} Desarrollaron [descripción] con mejoras de [porcentaje]%
    \item \textbf{\citet{autor2022_dominio1}:} Implementaron [descripción] resultando en [beneficios]
\end{itemize}

\subsection{[Dominio Específico 2]}

Los trabajos en [dominio específico] se han enfocado en [objetivos específicos]:

\begin{table}[htbp]
\centering
\caption{Resumen de trabajos en [Dominio Específico 2]}
\begin{tabular}{@{}p{3cm}p{4cm}p{3cm}p{3cm}@{}}
\toprule
\textbf{Autor/Año} & \textbf{Metodología} & \textbf{Dataset} & \textbf{Resultados} \\
\midrule
Autor et al. (2020) & CNN + LSTM & [Descripción] & Precisión: 89\% \\
Autor et al. (2021) & Ensemble Methods & [Descripción] & F1-Score: 0.92 \\
Autor et al. (2022) & Transformer & [Descripción] & RMSE: 0.15 \\
\bottomrule
\end{tabular}
\label{tab:trabajos_dominio2}
\end{table}

\section{Técnicas y Metodologías Específicas}

\subsection{[Técnica Específica 1]}

\subsubsection{Fundamentos}

La técnica [específica] fue introducida por \citet{autor_original} y se basa en [principios fundamentales]. Su formulación matemática es:

\begin{equation}
f(x) = \sum_{i=1}^{n} w_i \cdot \phi_i(x) + b
\label{eq:tecnica_especifica}
\end{equation}

donde $w_i$ son los pesos, $\phi_i(x)$ son las funciones de base, y $b$ es el sesgo.

\subsubsection{Desarrollos Recientes}

\textbf{Extensiones y Mejoras:}
\begin{itemize}
    \item \citet{autor2021_extension}: Propusieron una variación que [descripción de la mejora]
    \item \citet{autor2022_mejora}: Desarrollaron una versión optimizada que reduce [aspecto mejorado]
\end{itemize}

\textbf{Aplicaciones:}
\begin{itemize}
    \item Clasificación de [tipo de datos]: [Referencia y resultados]
    \item Predicción de [variable objetivo]: [Referencia y resultados]
    \item Optimización de [proceso]: [Referencia y resultados]
\end{itemize}

\subsection{[Técnica Específica 2]}

\subsubsection{Enfoques Híbridos}

La combinación de diferentes técnicas ha mostrado resultados prometedores:

\begin{itemize}
    \item \textbf{[Combinación 1]:} \citet{autor2023_hibrido1} combinaron [técnica A] con [técnica B] para [objetivo específico]
    \item \textbf{[Combinación 2]:} \citet{autor2023_hibrido2} integraron [técnica C] y [técnica D] logrando [resultados]
\end{itemize}

\section{Datasets y Benchmarks}

\subsection{Datasets Públicos Relevantes}

\begin{itemize}
    \item \textbf{[Nombre Dataset 1]:} 
        \begin{itemize}
            \item Fuente: [Organización/URL]
            \item Tamaño: [Número de instancias/características]
            \item Descripción: [Breve descripción]
            \item Trabajos que lo utilizan: \citet{autor1}, \citet{autor2}
        \end{itemize}
    
    \item \textbf{[Nombre Dataset 2]:}
        \begin{itemize}
            \item Fuente: [Organización/URL]
            \item Tamaño: [Número de instancias/características]
            \item Descripción: [Breve descripción]
            \item Trabajos que lo utilizan: \citet{autor3}, \citet{autor4}
        \end{itemize}
\end{itemize}

\subsection{Métricas de Evaluación Estándar}

Los trabajos revisados utilizan principalmente las siguientes métricas:

\begin{itemize}
    \item \textbf{Para Clasificación:} Precisión, Recall, F1-Score, AUC-ROC
    \item \textbf{Para Regresión:} RMSE, MAE, R²
    \item \textbf{Para Clustering:} Silhouette Score, Adjusted Rand Index
    \item \textbf{Métricas Específicas:} [Métricas específicas de tu dominio]
\end{itemize}

\section{Herramientas y Frameworks}

\subsection{Plataformas de Desarrollo}

\begin{table}[htbp]
\centering
\caption{Herramientas utilizadas en trabajos relacionados}
\begin{tabular}{@{}p{3cm}p{4cm}p{6cm}@{}}
\toprule
\textbf{Herramienta} & \textbf{Tipo} & \textbf{Trabajos que la utilizan} \\
\midrule
Python/Scikit-learn & ML Framework & \citet{autor1}, \citet{autor2}, \citet{autor3} \\
TensorFlow/Keras & Deep Learning & \citet{autor4}, \citet{autor5} \\
Apache Spark & Big Data & \citet{autor6}, \citet{autor7} \\
R/Caret & Statistical ML & \citet{autor8}, \citet{autor9} \\
\bottomrule
\end{tabular}
\label{tab:herramientas}
\end{table}

\section{Análisis Crítico y Brechas Identificadas}

\subsection{Fortalezas de los Enfoques Actuales}

\begin{enumerate}
    \item \textbf{Diversidad de Técnicas:} Existe una amplia gama de métodos disponibles
    \item \textbf{Resultados Prometedores:} Muchos trabajos reportan mejoras significativas
    \item \textbf{Validación Empírica:} La mayoría incluye evaluación experimental
    \item \textbf{Reproducibilidad:} Creciente tendencia hacia código abierto
\end{enumerate}

\subsection{Limitaciones y Brechas}

\begin{enumerate}
    \item \textbf{Escalabilidad:} Pocos trabajos abordan datasets de gran escala
    \item \textbf{Interpretabilidad:} Falta de explicabilidad en modelos complejos
    \item \textbf{Generalización:} Muchos estudios se limitan a datasets específicos
    \item \textbf{Comparación Justa:} Inconsistencias en evaluación experimental
    \item \textbf{Implementación Práctica:} Brecha entre investigación y aplicación real
\end{enumerate}

\subsection{Oportunidades de Investigación}

Basado en el análisis de la literatura, se identifican las siguientes oportunidades:

\begin{itemize}
    \item \textbf{[Oportunidad 1]:} [Descripción específica de la brecha u oportunidad]
    \item \textbf{[Oportunidad 2]:} [Descripción específica de la brecha u oportunidad]
    \item \textbf{[Oportunidad 3]:} [Descripción específica de la brecha u oportunidad]
\end{itemize}

\section{Posicionamiento de la Investigación}

\subsection{Diferenciación}

La presente investigación se diferencia de los trabajos existentes en:

\begin{enumerate}
    \item \textbf{Enfoque Metodológico:} [Describe cómo tu enfoque es diferente]
    \item \textbf{Conjunto de Datos:} [Explica las características únicas de tus datos]
    \item \textbf{Métricas de Evaluación:} [Menciona métricas adicionales o específicas]
    \item \textbf{Aplicación Práctica:} [Destaca el valor práctico de tu trabajo]
\end{enumerate}

\subsection{Contribuciones Esperadas}

Con base en las brechas identificadas, esta tesis contribuirá:

\begin{itemize}
    \item \textbf{Contribución Metodológica:} [Descripción específica]
    \item \textbf{Contribución Empírica:} [Descripción específica]
    \item \textbf{Contribución Práctica:} [Descripción específica]
\end{itemize}

\section{Conclusiones del Capítulo}

La revisión de la literatura revela que, aunque existe un cuerpo sustancial de investigación en [tu área específica], persisten brechas importantes en [áreas específicas]. Los trabajos existentes han logrado avances significativos en [aspectos específicos], pero presentan limitaciones en [aspectos específicos].

Esta investigación se posiciona para abordar específicamente [la brecha principal] mediante [tu enfoque propuesto], contribuyendo al estado del arte en [forma específica].

El siguiente capítulo presenta la metodología diseñada para abordar estas limitaciones y aprovechar las oportunidades identificadas.


% Capítulo 4: Metodología
% ========================================================================
% CAPÍTULO 4: METODOLOGÍA
% ========================================================================

\chapter{Metodología}

\section{Introducción}

Este capítulo describe la metodología utilizada para abordar el problema de investigación planteado. Se detalla el diseño experimental, las técnicas de analítica de datos empleadas, los conjuntos de datos utilizados, y los criterios de evaluación establecidos para validar los resultados.

\section{Enfoque Metodológico}

\subsection{Paradigma de Investigación}

Esta investigación sigue un paradigma \textbf{cuantitativo-experimental}, enfocándose en la medición objetiva del rendimiento de diferentes técnicas de analítica de datos aplicadas a [tu problema específico].

\subsection{Tipo de Investigación}

\begin{itemize}
    \item \textbf{Investigación Aplicada:} Orientada a resolver un problema práctico específico
    \item \textbf{Investigación Experimental:} Involucra manipulación controlada de variables
    \item \textbf{Investigación Cuantitativa:} Basada en mediciones numéricas y análisis estadístico
\end{itemize}

\section{Marco Metodológico General}

\subsection{Metodología CRISP-DM Adaptada}

El desarrollo de esta investigación sigue una adaptación de la metodología CRISP-DM, estructurada en las siguientes fases:

\begin{enumerate}
    \item \textbf{Comprensión del Problema}
        \begin{itemize}
            \item Definición clara del problema de negocio/investigación
            \item Identificación de objetivos y métricas de éxito
            \item Análisis de factibilidad técnica
        \end{itemize}
    
    \item \textbf{Comprensión de los Datos}
        \begin{itemize}
            \item Exploración inicial de datasets
            \item Análisis de calidad de datos
            \item Identificación de patrones preliminares
        \end{itemize}
    
    \item \textbf{Preparación de los Datos}
        \begin{itemize}
            \item Limpieza y preprocesamiento
            \item Transformación y normalización
            \item Ingeniería de características (Feature Engineering)
            \item División en conjuntos de entrenamiento/validación/prueba
        \end{itemize}
    
    \item \textbf{Modelado}
        \begin{itemize}
            \item Selección de algoritmos y técnicas
            \item Configuración de hiperparámetros
            \item Entrenamiento de modelos
            \item Validación cruzada
        \end{itemize}
    
    \item \textbf{Evaluación}
        \begin{itemize}
            \item Aplicación de métricas de rendimiento
            \item Análisis estadístico de resultados
            \item Comparación con métodos baseline
            \item Validación de hipótesis
        \end{itemize}
    
    \item \textbf{Implementación y Despliegue}
        \begin{itemize}
            \item Desarrollo de prototipo funcional
            \item Validación en entorno real (si aplica)
            \item Documentación de resultados
        \end{itemize}
\end{enumerate}

\section{Conjuntos de Datos}

\subsection{Fuentes de Datos}

\subsubsection{Dataset Principal: [Nombre del Dataset]}

\begin{itemize}
    \item \textbf{Fuente:} [Organización, URL o descripción de origen]
    \item \textbf{Tamaño:} [Número de instancias] registros, [Número de características] características
    \item \textbf{Período:} [Rango temporal de los datos]
    \item \textbf{Tipo de datos:} [Numéricos, categóricos, texto, imágenes, etc.]
    \item \textbf{Formato:} [CSV, JSON, Base de datos, etc.]
\end{itemize}

\textbf{Descripción de las variables:}

\begin{table}[htbp]
\centering
\caption{Variables del dataset principal}
\begin{tabular}{@{}p{3cm}p{2.5cm}p{6cm}p{2cm}@{}}
\toprule
\textbf{Variable} & \textbf{Tipo} & \textbf{Descripción} & \textbf{Rango} \\
\midrule
[Variable 1] & Numérico & [Descripción detallada] & [Min-Max] \\
[Variable 2] & Categórico & [Descripción detallada] & [Categorías] \\
[Variable 3] & Texto & [Descripción detallada] & [Longitud] \\
[Variable objetivo] & [Tipo] & [Descripción detallada] & [Valores] \\
\bottomrule
\end{tabular}
\label{tab:variables_dataset}
\end{table}

\subsubsection{Datasets Complementarios}

\begin{itemize}
    \item \textbf{Dataset Secundario 1:} [Descripción y propósito]
    \item \textbf{Dataset Secundario 2:} [Descripción y propósito]
    \item \textbf{Datos Externos:} [APIs, fuentes adicionales]
\end{itemize}

\subsection{Consideraciones Éticas y de Privacidad}

\begin{itemize}
    \item \textbf{Consentimiento:} [Descripción del consentimiento para uso de datos]
    \item \textbf{Anonimización:} [Técnicas utilizadas para proteger identidad]
    \item \textbf{Cumplimiento normativo:} [GDPR, regulaciones locales]
    \item \textbf{Sensibilidad:} [Manejo de información sensible]
\end{itemize}

\section{Preprocesamiento de Datos}

\subsection{Análisis Exploratorio de Datos (EDA)}

\subsubsection{Estadística Descriptiva}

\begin{itemize}
    \item Medidas de tendencia central y dispersión
    \item Análisis de distribuciones
    \item Identificación de valores atípicos
    \item Análisis de correlaciones
\end{itemize}

\begin{algorithm}[htbp]
\caption{Análisis Exploratorio de Datos}
\begin{algorithmic}[1]
\REQUIRE Dataset $D$ con variables $X = \{x_1, x_2, ..., x_n\}$
\ENSURE Reporte de estadísticas descriptivas y visualizaciones
\STATE Calcular estadísticas descriptivas para cada variable
\STATE Generar histogramas y gráficos de distribución
\STATE Calcular matriz de correlación
\STATE Identificar valores faltantes y atípicos
\STATE Crear visualizaciones de relaciones entre variables
\STATE Generar reporte de calidad de datos
\end{algorithmic}
\end{algorithm}

\subsubsection{Visualización de Datos}

\begin{itemize}
    \item Histogramas y gráficos de densidad
    \item Diagramas de caja (boxplots)
    \item Matrices de correlación
    \item Gráficos de dispersión
    \item Mapas de calor para variables categóricas
\end{itemize}

\subsection{Limpieza de Datos}

\subsubsection{Tratamiento de Valores Faltantes}

\begin{itemize}
    \item \textbf{Eliminación:} Para variables con >30\% de valores faltantes
    \item \textbf{Imputación simple:} Media/mediana para variables numéricas
    \item \textbf{Imputación múltiple:} Para patrones complejos de datos faltantes
    \item \textbf{Indicadores de missingness:} Variables dummy para valores faltantes
\end{itemize}

\begin{equation}
\text{Imputación por media: } \hat{x_i} = \frac{1}{n}\sum_{j=1}^{n} x_j
\label{eq:imputacion_media}
\end{equation}

\subsubsection{Detección y Tratamiento de Outliers}

\textbf{Método del Rango Intercuartílico (IQR):}
\begin{equation}
\text{Outlier si: } x < Q1 - 1.5 \times IQR \text{ o } x > Q3 + 1.5 \times IQR
\label{eq:outlier_iqr}
\end{equation}

\textbf{Método Z-Score:}
\begin{equation}
Z = \frac{x - \mu}{\sigma}, \text{ Outlier si } |Z| > 3
\label{eq:outlier_zscore}
\end{equation}

\subsection{Transformación de Datos}

\subsubsection{Normalización y Escalamiento}

\begin{itemize}
    \item \textbf{Min-Max Scaling:} 
    \begin{equation}
    x_{norm} = \frac{x - x_{min}}{x_{max} - x_{min}}
    \label{eq:minmax_scaling}
    \end{equation}
    
    \item \textbf{Z-Score Standardization:}
    \begin{equation}
    x_{std} = \frac{x - \mu}{\sigma}
    \label{eq:zscore_scaling}
    \end{equation}
    
    \item \textbf{Robust Scaling:}
    \begin{equation}
    x_{robust} = \frac{x - \text{mediana}}{IQR}
    \label{eq:robust_scaling}
    \end{equation}
\end{itemize}

\subsubsection{Codificación de Variables Categóricas}

\begin{itemize}
    \item \textbf{One-Hot Encoding:} Para variables nominales
    \item \textbf{Label Encoding:} Para variables ordinales
    \item \textbf{Target Encoding:} Para variables categóricas con alta cardinalidad
    \item \textbf{Binary Encoding:} Para reducir dimensionalidad
\end{itemize}

\subsection{Ingeniería de Características}

\subsubsection{Creación de Nuevas Variables}

\begin{itemize}
    \item \textbf{Variables de interacción:} Productos entre características existentes
    \item \textbf{Agregaciones temporales:} Para datos de series de tiempo
    \item \textbf{Características de dominio:} Basadas en conocimiento experto
    \item \textbf{Transformaciones matemáticas:} Log, raíz cuadrada, polinomiales
\end{itemize}

\subsubsection{Selección de Características}

\begin{itemize}
    \item \textbf{Métodos de filtro:} Correlación, pruebas chi-cuadrado, información mutua
    \item \textbf{Métodos wrapper:} Forward/backward selection, búsqueda exhaustiva
    \item \textbf{Métodos embedded:} LASSO, Random Forest feature importance
\end{itemize}

\section{Técnicas de Modelado}

\subsection{Algoritmos Seleccionados}

\subsubsection{Métodos Baseline}

\begin{itemize}
    \item \textbf{Regresión Lineal/Logística:} Modelo simple de referencia
    \item \textbf{Naive Bayes:} Para problemas de clasificación con independencia
    \item \textbf{k-NN:} Método no paramétrico simple
\end{itemize}

\subsubsection{Métodos Avanzados}

\begin{enumerate}
    \item \textbf{Support Vector Machines (SVM)}
        \begin{itemize}
            \item Kernels: lineal, polinomial, RBF
            \item Optimización de hiperparámetros C y gamma
            \item Aplicación para clasificación y regresión
        \end{itemize}
    
    \item \textbf{Random Forest}
        \begin{itemize}
            \item Número de árboles: 100-1000
            \item Profundidad máxima: optimización por validación cruzada
            \item Criterios de división: Gini/Entropy para clasificación
        \end{itemize}
    
    \item \textbf{Gradient Boosting (XGBoost/LightGBM)}
        \begin{itemize}
            \item Learning rate: 0.01-0.3
            \item Número de estimadores: búsqueda por grilla
            \item Regularización L1 y L2
        \end{itemize}
    
    \item \textbf{Redes Neuronales Profundas}
        \begin{itemize}
            \item Arquitecturas: MLP, CNN, RNN/LSTM
            \item Funciones de activación: ReLU, Sigmoid, Tanh
            \item Optimizadores: Adam, SGD, RMSprop
            \item Técnicas de regularización: Dropout, Batch Normalization
        \end{itemize}
\end{enumerate}

\subsection{Arquitecturas Específicas}

\subsubsection{[Modelo Propuesto/Principal]}

[Describe aquí la arquitectura específica de tu modelo principal, incluyendo:]

\begin{itemize}
    \item Diagrama de la arquitectura
    \item Componentes principales
    \item Función de pérdida utilizada
    \item Proceso de entrenamiento
\end{itemize}

\begin{algorithm}[htbp]
\caption{Algoritmo Principal Propuesto}
\begin{algorithmic}[1]
\REQUIRE Dataset de entrenamiento $D_{train}$, hiperparámetros $\theta$
\ENSURE Modelo entrenado $M$
\STATE Inicializar parámetros del modelo
\FOR{cada época $e$ desde 1 hasta $E$}
    \FOR{cada lote $B$ en $D_{train}$}
        \STATE Calcular predicciones $\hat{y} = f(X_B; \theta)$
        \STATE Calcular pérdida $L = loss(\hat{y}, y_B)$
        \STATE Calcular gradientes $\nabla_\theta L$
        \STATE Actualizar parámetros $\theta = \theta - \alpha \nabla_\theta L$
    \ENDFOR
    \STATE Evaluar en conjunto de validación
    \IF{criterio de parada se cumple}
        \STATE \textbf{break}
    \ENDIF
\ENDFOR
\RETURN Modelo $M$ con parámetros $\theta$
\end{algorithmic}
\end{algorithm}

\section{Diseño Experimental}

\subsection{División de Datos}

\begin{itemize}
    \item \textbf{Entrenamiento:} 70\% de los datos
    \item \textbf{Validación:} 15\% de los datos (para optimización de hiperparámetros)
    \item \textbf{Prueba:} 15\% de los datos (para evaluación final)
\end{itemize}

\textbf{Estrategia de división:}
\begin{itemize}
    \item División estratificada para mantener distribución de clases
    \item División temporal para datos de series de tiempo
    \item Validación cruzada k-fold (k=5) para robustez
\end{itemize}

\subsection{Optimización de Hiperparámetros}

\subsubsection{Técnicas de Búsqueda}

\begin{enumerate}
    \item \textbf{Grid Search:} Búsqueda exhaustiva en grilla predefinida
    \item \textbf{Random Search:} Búsqueda aleatoria en espacio de parámetros
    \item \textbf{Bayesian Optimization:} Optimización bayesiana con Gaussian Processes
    \item \textbf{Evolutionary Algorithms:} Algoritmos genéticos para espacios complejos
\end{enumerate}

\subsubsection{Espacios de Búsqueda}

\begin{table}[htbp]
\centering
\caption{Rangos de hiperparámetros por algoritmo}
\begin{tabular}{@{}p{3cm}p{4cm}p{6cm}@{}}
\toprule
\textbf{Algoritmo} & \textbf{Parámetro} & \textbf{Rango de Búsqueda} \\
\midrule
Random Forest & n\_estimators & [50, 100, 200, 500, 1000] \\
 & max\_depth & [3, 5, 10, 15, None] \\
 & min\_samples\_split & [2, 5, 10, 20] \\
\midrule
XGBoost & learning\_rate & [0.01, 0.05, 0.1, 0.2, 0.3] \\
 & max\_depth & [3, 4, 5, 6, 7, 8] \\
 & n\_estimators & [100, 200, 500, 1000] \\
\midrule
SVM & C & [0.1, 1, 10, 100, 1000] \\
 & gamma & [0.001, 0.01, 0.1, 1, auto] \\
 & kernel & [linear, rbf, poly] \\
\bottomrule
\end{tabular}
\label{tab:hiperparametros}
\end{table}

\section{Métricas de Evaluación}

\subsection{Métricas Primarias}

Para problemas de \textbf{clasificación}:
\begin{itemize}
    \item \textbf{Accuracy:} $Acc = \frac{TP + TN}{TP + TN + FP + FN}$
    \item \textbf{Precision:} $P = \frac{TP}{TP + FP}$
    \item \textbf{Recall:} $R = \frac{TP}{TP + FN}$
    \item \textbf{F1-Score:} $F1 = \frac{2 \times P \times R}{P + R}$
    \item \textbf{AUC-ROC:} Área bajo la curva ROC
\end{itemize}

Para problemas de \textbf{regresión}:
\begin{itemize}
    \item \textbf{RMSE:} $RMSE = \sqrt{\frac{1}{n}\sum_{i=1}^{n}(y_i - \hat{y_i})^2}$
    \item \textbf{MAE:} $MAE = \frac{1}{n}\sum_{i=1}^{n}|y_i - \hat{y_i}|$
    \item \textbf{R²:} $R^2 = 1 - \frac{\sum_{i=1}^{n}(y_i - \hat{y_i})^2}{\sum_{i=1}^{n}(y_i - \bar{y})^2}$
    \item \textbf{MAPE:} $MAPE = \frac{100\%}{n}\sum_{i=1}^{n}\left|\frac{y_i - \hat{y_i}}{y_i}\right|$
\end{itemize}

\subsection{Métricas Secundarias}

\begin{itemize}
    \item \textbf{Tiempo de entrenamiento:} Eficiencia computacional
    \item \textbf{Tiempo de inferencia:} Velocidad de predicción
    \item \textbf{Memoria utilizada:} Recursos computacionales
    \item \textbf{Interpretabilidad:} Métricas de explicabilidad (si aplica)
\end{itemize}

\section{Validación Estadística}

\subsection{Pruebas de Significancia}

\begin{itemize}
    \item \textbf{Prueba t de Student:} Comparación de medias entre dos modelos
    \item \textbf{ANOVA:} Comparación múltiple de modelos
    \item \textbf{Prueba de Wilcoxon:} Para distribuciones no normales
    \item \textbf{Corrección de Bonferroni:} Para comparaciones múltiples
\end{itemize}

\subsection{Intervalos de Confianza}

Cálculo de intervalos de confianza del 95\% para las métricas principales:
\begin{equation}
IC_{95\%} = \bar{x} \pm t_{n-1,0.025} \times \frac{s}{\sqrt{n}}
\label{eq:intervalo_confianza}
\end{equation}

\section{Herramientas y Tecnologías}

\subsection{Entorno de Desarrollo}

\begin{itemize}
    \item \textbf{Lenguaje principal:} Python 3.8+
    \item \textbf{IDE:} Jupyter Notebook, PyCharm, VS Code
    \item \textbf{Control de versiones:} Git/GitHub
    \item \textbf{Gestión de dependencias:} Conda/pip
\end{itemize}

\subsection{Librerías y Frameworks}

\begin{table}[htbp]
\centering
\caption{Principales librerías utilizadas}
\begin{tabular}{@{}p{3cm}p{3cm}p{7cm}@{}}
\toprule
\textbf{Categoría} & \textbf{Librería} & \textbf{Propósito} \\
\midrule
Manipulación de datos & Pandas & Análisis y manipulación de datos \\
 & NumPy & Operaciones numéricas \\
\midrule
Machine Learning & Scikit-learn & Algoritmos tradicionales de ML \\
 & XGBoost & Gradient boosting \\
 & LightGBM & Gradient boosting eficiente \\
\midrule
Deep Learning & TensorFlow/Keras & Redes neuronales profundas \\
 & PyTorch & Investigación en deep learning \\
\midrule
Visualización & Matplotlib & Gráficos básicos \\
 & Seaborn & Visualización estadística \\
 & Plotly & Gráficos interactivos \\
\midrule
Estadística & SciPy & Pruebas estadísticas \\
 & Statsmodels & Modelos estadísticos \\
\bottomrule
\end{tabular}
\label{tab:librerias}
\end{table}

\subsection{Infraestructura Computacional}

\begin{itemize}
    \item \textbf{Hardware local:} [Especificaciones del equipo]
    \item \textbf{Cloud computing:} [AWS/Azure/GCP si aplica]
    \item \textbf{GPUs:} [Para entrenamiento de deep learning]
    \item \textbf{Almacenamiento:} [Bases de datos, sistemas de archivos]
\end{itemize}

\section{Reproducibilidad}

\subsection{Semillas Aleatorias}

Para garantizar reproducibilidad:
\begin{itemize}
    \item Fijación de semillas aleatorias en todas las librerías
    \item Documentación de versiones de software
    \item Registro de configuraciones de hardware
\end{itemize}

\subsection{Documentación}

\begin{itemize}
    \item Código comentado y documentado
    \item Notebooks con resultados ejecutados
    \item Scripts de automatización
    \item Archivos de configuración versionados
\end{itemize}

\section{Cronograma de Actividades}

\begin{table}[htbp]
\centering
\caption{Cronograma de desarrollo metodológico}
\begin{tabular}{@{}p{4cm}p{2cm}p{2cm}p{2cm}p{2cm}@{}}
\toprule
\textbf{Actividad} & \textbf{Mes 1} & \textbf{Mes 2} & \textbf{Mes 3} & \textbf{Mes 4} \\
\midrule
Recolección de datos & X & & & \\
Análisis exploratorio & X & X & & \\
Preprocesamiento & & X & X & \\
Modelado baseline & & & X & \\
Modelado avanzado & & & X & X \\
Evaluación y validación & & & & X \\
Documentación & X & X & X & X \\
\bottomrule
\end{tabular}
\label{tab:cronograma}
\end{table}

\section{Consideraciones Éticas}

\subsection{Privacidad de Datos}

\begin{itemize}
    \item Anonimización de datos personales
    \item Cumplimiento de regulaciones de privacidad
    \item Acceso controlado a datos sensibles
    \item Eliminación segura de datos temporales
\end{itemize}

\subsection{Sesgo y Fairness}

\begin{itemize}
    \item Análisis de sesgo en datos de entrenamiento
    \item Evaluación de fairness en diferentes grupos demográficos
    \item Métricas de equidad algorítmica
    \item Mitigación de discriminación algorítmica
\end{itemize}

\section{Conclusiones del Capítulo}

Este capítulo ha presentado una metodología comprensiva que combina técnicas tradicionales y modernas de analítica de datos para abordar [el problema específico]. La metodología propuesta está diseñada para ser rigurosa, reproducible y éticamente responsable.

El siguiente capítulo presenta la implementación detallada de esta metodología y los resultados obtenidos en cada fase del proceso.


% Capítulo 5: Desarrollo/Implementación
% ========================================================================
% CAPÍTULO 5: DESARROLLO E IMPLEMENTACIÓN
% ========================================================================

\chapter{Desarrollo e Implementación}

\section{Introducción}

Este capítulo presenta la implementación práctica de la metodología descrita en el capítulo anterior. Se detallan los procesos de desarrollo, las decisiones técnicas tomadas, la arquitectura del sistema implementado y los desafíos encontrados durante la fase de desarrollo.

\section{Arquitectura del Sistema}

\subsection{Diseño General}

La solución implementada sigue una arquitectura modular compuesta por los siguientes componentes principales:

\begin{enumerate}
    \item \textbf{Módulo de Ingesta de Datos}
    \item \textbf{Módulo de Preprocesamiento}
    \item \textbf{Módulo de Entrenamiento de Modelos}
    \item \textbf{Módulo de Evaluación}
    \item \textbf{Módulo de Visualización y Reportes}
    \item \textbf{API de Predicción}
\end{enumerate}

\begin{figure}[htbp]
\centering
% \includegraphics[width=0.9\textwidth]{imagenes/arquitectura_sistema.png}
\caption{Arquitectura general del sistema implementado}
\label{fig:arquitectura_sistema}
\end{figure}

\subsection{Tecnologías Utilizadas}

\begin{table}[htbp]
\centering
\caption{Stack tecnológico implementado}
\begin{tabular}{@{}p{3cm}p{4cm}p{6cm}@{}}
\toprule
\textbf{Componente} & \textbf{Tecnología} & \textbf{Justificación} \\
\midrule
Backend & Python 3.9 & Ecosistema robusto para ML/Data Science \\
Manipulación de datos & Pandas, NumPy & Eficiencia en procesamiento de datos \\
Machine Learning & Scikit-learn, XGBoost & Algoritmos optimizados y probados \\
Deep Learning & TensorFlow 2.8 & Flexibilidad y escalabilidad \\
Visualización & Matplotlib, Plotly & Gráficos estáticos e interactivos \\
API & FastAPI & Alto rendimiento y documentación automática \\
Base de datos & PostgreSQL & Robustez para datos relacionales \\
Contenedores & Docker & Reproducibilidad y despliegue \\
Orquestación & Docker Compose & Gestión de servicios múltiples \\
\bottomrule
\end{tabular}
\label{tab:stack_tecnologico}
\end{table}

\section{Implementación del Pipeline de Datos}

\subsection{Módulo de Ingesta de Datos}

\subsubsection{Conectores de Datos}

Se implementaron conectores para múltiples fuentes de datos:

\begin{lstlisting}[language=Python, caption=Conector base para fuentes de datos]
from abc import ABC, abstractmethod
import pandas as pd
from typing import Dict, Any

class DataConnector(ABC):
    """Clase base para conectores de datos"""
    
    def __init__(self, config: Dict[str, Any]):
        self.config = config
        self.connection = None
    
    @abstractmethod
    def connect(self) -> bool:
        """Establece conexión con la fuente de datos"""
        pass
    
    @abstractmethod
    def fetch_data(self, query: str = None) -> pd.DataFrame:
        """Extrae datos de la fuente"""
        pass
    
    @abstractmethod
    def disconnect(self) -> None:
        """Cierra la conexión"""
        pass

class CSVConnector(DataConnector):
    """Conector para archivos CSV"""
    
    def connect(self) -> bool:
        return True
    
    def fetch_data(self, file_path: str = None) -> pd.DataFrame:
        file_path = file_path or self.config.get('file_path')
        return pd.read_csv(file_path)
    
    def disconnect(self) -> None:
        pass

class DatabaseConnector(DataConnector):
    """Conector para bases de datos relacionales"""
    
    def connect(self) -> bool:
        # Implementación de conexión a BD
        pass
    
    def fetch_data(self, query: str) -> pd.DataFrame:
        # Implementación de consulta SQL
        pass
    
    def disconnect(self) -> None:
        # Cierre de conexión
        pass
\end{lstlisting}

\subsubsection{Validación de Datos}

\begin{lstlisting}[language=Python, caption=Sistema de validación de datos]
import pandas as pd
from typing import List, Dict, Tuple
import logging

class DataValidator:
    """Validador de calidad de datos"""
    
    def __init__(self):
        self.validation_rules = {}
        self.logger = logging.getLogger(__name__)
    
    def add_rule(self, column: str, rule_type: str, **kwargs):
        """Añade regla de validación para una columna"""
        if column not in self.validation_rules:
            self.validation_rules[column] = []
        
        self.validation_rules[column].append({
            'type': rule_type,
            'params': kwargs
        })
    
    def validate_dataset(self, df: pd.DataFrame) -> Dict[str, List[str]]:
        """Valida el dataset completo"""
        validation_results = {
            'errors': [],
            'warnings': [],
            'info': []
        }
        
        # Validaciones generales
        validation_results['info'].append(
            f"Dataset shape: {df.shape}"
        )
        
        # Validar valores faltantes
        missing_values = df.isnull().sum()
        for col, missing_count in missing_values.items():
            if missing_count > 0:
                missing_pct = (missing_count / len(df)) * 100
                if missing_pct > 30:
                    validation_results['errors'].append(
                        f"Column '{col}' has {missing_pct:.1f}% missing values"
                    )
                elif missing_pct > 10:
                    validation_results['warnings'].append(
                        f"Column '{col}' has {missing_pct:.1f}% missing values"
                    )
        
        # Validar duplicados
        duplicates = df.duplicated().sum()
        if duplicates > 0:
            validation_results['warnings'].append(
                f"Found {duplicates} duplicate rows"
            )
        
        # Aplicar reglas específicas por columna
        for column, rules in self.validation_rules.items():
            if column in df.columns:
                for rule in rules:
                    result = self._apply_rule(df[column], rule)
                    if result:
                        validation_results['errors'].append(result)
        
        return validation_results
    
    def _apply_rule(self, series: pd.Series, rule: Dict) -> str:
        """Aplica una regla específica de validación"""
        rule_type = rule['type']
        params = rule['params']
        
        if rule_type == 'range':
            min_val, max_val = params['min'], params['max']
            violations = ((series < min_val) | (series > max_val)).sum()
            if violations > 0:
                return f"Column '{series.name}' has {violations} values outside range [{min_val}, {max_val}]"
        
        elif rule_type == 'categorical':
            allowed_values = set(params['values'])
            invalid_values = set(series.unique()) - allowed_values
            if invalid_values:
                return f"Column '{series.name}' has invalid values: {invalid_values}"
        
        return None
\end{lstlisting}

\subsection{Módulo de Preprocesamiento}

\subsubsection{Pipeline de Transformaciones}

\begin{lstlisting}[language=Python, caption=Pipeline de preprocesamiento modular]
from sklearn.base import BaseEstimator, TransformerMixin
from sklearn.pipeline import Pipeline
from sklearn.compose import ColumnTransformer
from sklearn.preprocessing import StandardScaler, OneHotEncoder, LabelEncoder
from sklearn.impute import SimpleImputer
import numpy as np

class OutlierRemover(BaseEstimator, TransformerMixin):
    """Transformer para remover outliers usando IQR"""
    
    def __init__(self, factor=1.5):
        self.factor = factor
        self.bounds_ = {}
    
    def fit(self, X, y=None):
        for col in X.columns:
            if X[col].dtype in ['int64', 'float64']:
                Q1 = X[col].quantile(0.25)
                Q3 = X[col].quantile(0.75)
                IQR = Q3 - Q1
                lower_bound = Q1 - self.factor * IQR
                upper_bound = Q3 + self.factor * IQR
                self.bounds_[col] = (lower_bound, upper_bound)
        return self
    
    def transform(self, X):
        X_clean = X.copy()
        for col, (lower, upper) in self.bounds_.items():
            if col in X_clean.columns:
                X_clean = X_clean[
                    (X_clean[col] >= lower) & (X_clean[col] <= upper)
                ]
        return X_clean

class FeatureEngineer(BaseEstimator, TransformerMixin):
    """Transformer para ingeniería de características"""
    
    def __init__(self, create_interactions=True, polynomial_degree=2):
        self.create_interactions = create_interactions
        self.polynomial_degree = polynomial_degree
        self.feature_names_ = []
    
    def fit(self, X, y=None):
        self.feature_names_ = list(X.columns)
        return self
    
    def transform(self, X):
        X_engineered = X.copy()
        
        # Crear características polinomiales
        numeric_cols = X.select_dtypes(include=[np.number]).columns
        for col in numeric_cols:
            if self.polynomial_degree >= 2:
                X_engineered[f"{col}_squared"] = X[col] ** 2
            if self.polynomial_degree >= 3:
                X_engineered[f"{col}_cubed"] = X[col] ** 3
        
        # Crear interacciones entre variables numéricas
        if self.create_interactions and len(numeric_cols) > 1:
            for i, col1 in enumerate(numeric_cols):
                for col2 in numeric_cols[i+1:]:
                    X_engineered[f"{col1}_{col2}_interaction"] = (
                        X[col1] * X[col2]
                    )
        
        return X_engineered

def create_preprocessing_pipeline(numeric_features, categorical_features):
    """Crea pipeline de preprocesamiento completo"""
    
    # Pipeline para características numéricas
    numeric_pipeline = Pipeline([
        ('imputer', SimpleImputer(strategy='median')),
        ('scaler', StandardScaler())
    ])
    
    # Pipeline para características categóricas
    categorical_pipeline = Pipeline([
        ('imputer', SimpleImputer(strategy='constant', fill_value='missing')),
        ('onehot', OneHotEncoder(handle_unknown='ignore', sparse=False))
    ])
    
    # Combinar pipelines
    preprocessor = ColumnTransformer([
        ('num', numeric_pipeline, numeric_features),
        ('cat', categorical_pipeline, categorical_features)
    ])
    
    # Pipeline completo
    full_pipeline = Pipeline([
        ('outlier_removal', OutlierRemover()),
        ('feature_engineering', FeatureEngineer()),
        ('preprocessing', preprocessor)
    ])
    
    return full_pipeline
\end{lstlisting}

\section{Implementación de Modelos}

\subsection{Clase Base para Modelos}

\begin{lstlisting}[language=Python, caption=Arquitectura base para modelos de ML]
from abc import ABC, abstractmethod
from typing import Dict, Any, Tuple
import joblib
import pandas as pd
import numpy as np
from sklearn.metrics import classification_report, mean_squared_error
import logging

class BaseModel(ABC):
    """Clase base para todos los modelos de ML"""
    
    def __init__(self, name: str, config: Dict[str, Any]):
        self.name = name
        self.config = config
        self.model = None
        self.is_fitted = False
        self.feature_names = None
        self.logger = logging.getLogger(f"{__name__}.{name}")
    
    @abstractmethod
    def build_model(self) -> Any:
        """Construye el modelo con la configuración dada"""
        pass
    
    @abstractmethod
    def train(self, X_train: pd.DataFrame, y_train: pd.Series, 
              X_val: pd.DataFrame = None, y_val: pd.Series = None) -> Dict[str, Any]:
        """Entrena el modelo"""
        pass
    
    def predict(self, X: pd.DataFrame) -> np.ndarray:
        """Realiza predicciones"""
        if not self.is_fitted:
            raise ValueError("Model must be fitted before making predictions")
        
        return self.model.predict(X)
    
    def predict_proba(self, X: pd.DataFrame) -> np.ndarray:
        """Realiza predicciones de probabilidad (solo para clasificación)"""
        if not self.is_fitted:
            raise ValueError("Model must be fitted before making predictions")
        
        if hasattr(self.model, 'predict_proba'):
            return self.model.predict_proba(X)
        else:
            raise NotImplementedError("Model does not support probability predictions")
    
    def save_model(self, filepath: str) -> None:
        """Guarda el modelo entrenado"""
        model_data = {
            'model': self.model,
            'config': self.config,
            'feature_names': self.feature_names,
            'is_fitted': self.is_fitted
        }
        joblib.dump(model_data, filepath)
        self.logger.info(f"Model saved to {filepath}")
    
    def load_model(self, filepath: str) -> None:
        """Carga un modelo previamente entrenado"""
        model_data = joblib.load(filepath)
        self.model = model_data['model']
        self.config = model_data['config']
        self.feature_names = model_data['feature_names']
        self.is_fitted = model_data['is_fitted']
        self.logger.info(f"Model loaded from {filepath}")
    
    def get_feature_importance(self) -> pd.DataFrame:
        """Obtiene la importancia de las características"""
        if not self.is_fitted:
            raise ValueError("Model must be fitted first")
        
        if hasattr(self.model, 'feature_importances_'):
            importance_df = pd.DataFrame({
                'feature': self.feature_names,
                'importance': self.model.feature_importances_
            }).sort_values('importance', ascending=False)
            return importance_df
        else:
            raise NotImplementedError("Model does not provide feature importance")

class RandomForestModel(BaseModel):
    """Implementación de Random Forest"""
    
    def build_model(self):
        from sklearn.ensemble import RandomForestClassifier, RandomForestRegressor
        
        task_type = self.config.get('task_type', 'classification')
        model_params = self.config.get('model_params', {})
        
        if task_type == 'classification':
            self.model = RandomForestClassifier(**model_params)
        else:
            self.model = RandomForestRegressor(**model_params)
        
        return self.model
    
    def train(self, X_train: pd.DataFrame, y_train: pd.Series, 
              X_val: pd.DataFrame = None, y_val: pd.Series = None) -> Dict[str, Any]:
        
        if self.model is None:
            self.build_model()
        
        self.feature_names = list(X_train.columns)
        
        # Entrenar el modelo
        self.logger.info("Starting training...")
        self.model.fit(X_train, y_train)
        self.is_fitted = True
        
        # Evaluar en conjunto de entrenamiento
        train_score = self.model.score(X_train, y_train)
        results = {'train_score': train_score}
        
        # Evaluar en conjunto de validación si está disponible
        if X_val is not None and y_val is not None:
            val_score = self.model.score(X_val, y_val)
            results['val_score'] = val_score
            self.logger.info(f"Validation score: {val_score:.4f}")
        
        self.logger.info("Training completed")
        return results

class XGBoostModel(BaseModel):
    """Implementación de XGBoost"""
    
    def build_model(self):
        import xgboost as xgb
        
        task_type = self.config.get('task_type', 'classification')
        model_params = self.config.get('model_params', {})
        
        if task_type == 'classification':
            self.model = xgb.XGBClassifier(**model_params)
        else:
            self.model = xgb.XGBRegressor(**model_params)
        
        return self.model
    
    def train(self, X_train: pd.DataFrame, y_train: pd.Series, 
              X_val: pd.DataFrame = None, y_val: pd.Series = None) -> Dict[str, Any]:
        
        if self.model is None:
            self.build_model()
        
        self.feature_names = list(X_train.columns)
        
        # Configurar early stopping si hay datos de validación
        fit_params = {}
        if X_val is not None and y_val is not None:
            fit_params['eval_set'] = [(X_val, y_val)]
            fit_params['early_stopping_rounds'] = 10
            fit_params['verbose'] = False
        
        # Entrenar el modelo
        self.logger.info("Starting XGBoost training...")
        self.model.fit(X_train, y_train, **fit_params)
        self.is_fitted = True
        
        # Evaluar
        train_score = self.model.score(X_train, y_train)
        results = {'train_score': train_score}
        
        if X_val is not None and y_val is not None:
            val_score = self.model.score(X_val, y_val)
            results['val_score'] = val_score
        
        self.logger.info("XGBoost training completed")
        return results
\end{lstlisting}

\subsection{Implementación de Red Neuronal}

\begin{lstlisting}[language=Python, caption=Implementación de red neuronal con TensorFlow]
import tensorflow as tf
from tensorflow import keras
from tensorflow.keras import layers
import numpy as np
import pandas as pd

class NeuralNetworkModel(BaseModel):
    """Implementación de Red Neuronal con TensorFlow"""
    
    def build_model(self):
        """Construye la arquitectura de la red neuronal"""
        input_dim = self.config.get('input_dim', 10)
        hidden_layers = self.config.get('hidden_layers', [64, 32])
        output_dim = self.config.get('output_dim', 1)
        task_type = self.config.get('task_type', 'classification')
        dropout_rate = self.config.get('dropout_rate', 0.3)
        
        # Crear el modelo secuencial
        model = keras.Sequential()
        
        # Capa de entrada
        model.add(layers.Dense(hidden_layers[0], 
                              input_dim=input_dim, 
                              activation='relu'))
        model.add(layers.Dropout(dropout_rate))
        
        # Capas ocultas
        for hidden_size in hidden_layers[1:]:
            model.add(layers.Dense(hidden_size, activation='relu'))
            model.add(layers.Dropout(dropout_rate))
        
        # Capa de salida
        if task_type == 'classification':
            if output_dim == 1:
                model.add(layers.Dense(1, activation='sigmoid'))
                loss = 'binary_crossentropy'
                metrics = ['accuracy']
            else:
                model.add(layers.Dense(output_dim, activation='softmax'))
                loss = 'categorical_crossentropy'
                metrics = ['accuracy']
        else:  # regression
            model.add(layers.Dense(output_dim, activation='linear'))
            loss = 'mse'
            metrics = ['mae']
        
        # Compilar el modelo
        optimizer = self.config.get('optimizer', 'adam')
        learning_rate = self.config.get('learning_rate', 0.001)
        
        if optimizer == 'adam':
            opt = keras.optimizers.Adam(learning_rate=learning_rate)
        elif optimizer == 'sgd':
            opt = keras.optimizers.SGD(learning_rate=learning_rate)
        else:
            opt = optimizer
        
        model.compile(optimizer=opt, loss=loss, metrics=metrics)
        
        self.model = model
        return model
    
    def train(self, X_train: pd.DataFrame, y_train: pd.Series, 
              X_val: pd.DataFrame = None, y_val: pd.Series = None) -> Dict[str, Any]:
        
        if self.model is None:
            # Actualizar input_dim basado en los datos reales
            self.config['input_dim'] = X_train.shape[1]
            self.build_model()
        
        self.feature_names = list(X_train.columns)
        
        # Configurar callbacks
        callbacks = []
        
        # Early stopping
        if X_val is not None:
            early_stopping = keras.callbacks.EarlyStopping(
                monitor='val_loss',
                patience=10,
                restore_best_weights=True
            )
            callbacks.append(early_stopping)
        
        # Model checkpoint
        checkpoint = keras.callbacks.ModelCheckpoint(
            f'models/best_{self.name}_model.h5',
            monitor='val_loss' if X_val is not None else 'loss',
            save_best_only=True
        )
        callbacks.append(checkpoint)
        
        # Configurar datos de validación
        validation_data = None
        if X_val is not None and y_val is not None:
            validation_data = (X_val.values, y_val.values)
        
        # Entrenar el modelo
        epochs = self.config.get('epochs', 100)
        batch_size = self.config.get('batch_size', 32)
        
        self.logger.info("Starting neural network training...")
        history = self.model.fit(
            X_train.values, y_train.values,
            epochs=epochs,
            batch_size=batch_size,
            validation_data=validation_data,
            callbacks=callbacks,
            verbose=1
        )
        
        self.is_fitted = True
        self.logger.info("Neural network training completed")
        
        # Preparar resultados
        results = {
            'history': history.history,
            'final_loss': history.history['loss'][-1]
        }
        
        if validation_data is not None:
            results['final_val_loss'] = history.history['val_loss'][-1]
        
        return results
    
    def predict(self, X: pd.DataFrame) -> np.ndarray:
        """Realiza predicciones"""
        if not self.is_fitted:
            raise ValueError("Model must be fitted before making predictions")
        
        predictions = self.model.predict(X.values)
        
        # Para clasificación binaria, convertir probabilidades a clases
        task_type = self.config.get('task_type', 'classification')
        if task_type == 'classification' and predictions.shape[1] == 1:
            predictions = (predictions > 0.5).astype(int)
        
        return predictions.flatten()
\end{lstlisting}

\section{Sistema de Evaluación}

\subsection{Módulo de Métricas}

\begin{lstlisting}[language=Python, caption=Sistema completo de evaluación]
from sklearn.metrics import (
    accuracy_score, precision_score, recall_score, f1_score,
    roc_auc_score, confusion_matrix, classification_report,
    mean_squared_error, mean_absolute_error, r2_score
)
import matplotlib.pyplot as plt
import seaborn as sns
import numpy as np
import pandas as pd
from typing import Dict, List, Any

class ModelEvaluator:
    """Evaluador completo de modelos de ML"""
    
    def __init__(self, task_type: str = 'classification'):
        self.task_type = task_type
        self.results = {}
    
    def evaluate_model(self, model, X_test: pd.DataFrame, y_test: pd.Series, 
                      model_name: str) -> Dict[str, Any]:
        """Evalúa un modelo y almacena los resultados"""
        
        predictions = model.predict(X_test)
        
        if self.task_type == 'classification':
            metrics = self._evaluate_classification(y_test, predictions, model, X_test)
        else:
            metrics = self._evaluate_regression(y_test, predictions)
        
        # Agregar información del modelo
        metrics['model_name'] = model_name
        metrics['n_samples'] = len(y_test)
        metrics['n_features'] = X_test.shape[1]
        
        self.results[model_name] = metrics
        return metrics
    
    def _evaluate_classification(self, y_true: pd.Series, y_pred: np.ndarray, 
                               model, X_test: pd.DataFrame) -> Dict[str, Any]:
        """Evalúa modelo de clasificación"""
        
        metrics = {}
        
        # Métricas básicas
        metrics['accuracy'] = accuracy_score(y_true, y_pred)
        metrics['precision'] = precision_score(y_true, y_pred, average='weighted')
        metrics['recall'] = recall_score(y_true, y_pred, average='weighted')
        metrics['f1_score'] = f1_score(y_true, y_pred, average='weighted')
        
        # AUC-ROC (si el modelo soporta predict_proba)
        try:
            y_proba = model.predict_proba(X_test)
            if y_proba.shape[1] == 2:  # Clasificación binaria
                metrics['auc_roc'] = roc_auc_score(y_true, y_proba[:, 1])
            else:  # Multiclase
                metrics['auc_roc'] = roc_auc_score(y_true, y_proba, 
                                                 multi_class='ovr', average='weighted')
        except:
            metrics['auc_roc'] = None
        
        # Matriz de confusión
        metrics['confusion_matrix'] = confusion_matrix(y_true, y_pred)
        
        # Reporte de clasificación
        metrics['classification_report'] = classification_report(
            y_true, y_pred, output_dict=True
        )
        
        return metrics
    
    def _evaluate_regression(self, y_true: pd.Series, y_pred: np.ndarray) -> Dict[str, Any]:
        """Evalúa modelo de regresión"""
        
        metrics = {}
        
        # Métricas de regresión
        metrics['mse'] = mean_squared_error(y_true, y_pred)
        metrics['rmse'] = np.sqrt(metrics['mse'])
        metrics['mae'] = mean_absolute_error(y_true, y_pred)
        metrics['r2'] = r2_score(y_true, y_pred)
        
        # MAPE (Mean Absolute Percentage Error)
        mape = np.mean(np.abs((y_true - y_pred) / y_true)) * 100
        metrics['mape'] = mape
        
        # Errores residuales
        residuals = y_true - y_pred
        metrics['residuals_mean'] = np.mean(residuals)
        metrics['residuals_std'] = np.std(residuals)
        
        return metrics
    
    def compare_models(self) -> pd.DataFrame:
        """Compara todos los modelos evaluados"""
        
        if not self.results:
            return pd.DataFrame()
        
        comparison_data = []
        
        for model_name, metrics in self.results.items():
            row = {'Model': model_name}
            
            if self.task_type == 'classification':
                row.update({
                    'Accuracy': metrics['accuracy'],
                    'Precision': metrics['precision'],
                    'Recall': metrics['recall'],
                    'F1-Score': metrics['f1_score'],
                    'AUC-ROC': metrics.get('auc_roc', 'N/A')
                })
            else:
                row.update({
                    'RMSE': metrics['rmse'],
                    'MAE': metrics['mae'],
                    'R²': metrics['r2'],
                    'MAPE': metrics['mape']
                })
            
            comparison_data.append(row)
        
        comparison_df = pd.DataFrame(comparison_data)
        return comparison_df.sort_values(
            'Accuracy' if self.task_type == 'classification' else 'R²', 
            ascending=False
        )
    
    def plot_comparison(self, metric: str = None, save_path: str = None):
        """Crea gráfico de comparación entre modelos"""
        
        if not self.results:
            print("No hay resultados para comparar")
            return
        
        comparison_df = self.compare_models()
        
        if metric is None:
            metric = 'Accuracy' if self.task_type == 'classification' else 'R²'
        
        if metric not in comparison_df.columns:
            print(f"Métrica '{metric}' no disponible")
            return
        
        plt.figure(figsize=(10, 6))
        bars = plt.bar(comparison_df['Model'], comparison_df[metric])
        
        # Añadir valores sobre las barras
        for bar in bars:
            height = bar.get_height()
            if isinstance(height, (int, float)):
                plt.text(bar.get_x() + bar.get_width()/2., height,
                        f'{height:.3f}',
                        ha='center', va='bottom')
        
        plt.title(f'Comparación de Modelos - {metric}')
        plt.xlabel('Modelo')
        plt.ylabel(metric)
        plt.xticks(rotation=45)
        plt.tight_layout()
        
        if save_path:
            plt.savefig(save_path, dpi=300, bbox_inches='tight')
        
        plt.show()
    
    def plot_confusion_matrix(self, model_name: str, save_path: str = None):
        """Grafica matriz de confusión para modelo de clasificación"""
        
        if self.task_type != 'classification':
            print("Matriz de confusión solo disponible para clasificación")
            return
        
        if model_name not in self.results:
            print(f"Modelo '{model_name}' no encontrado")
            return
        
        cm = self.results[model_name]['confusion_matrix']
        
        plt.figure(figsize=(8, 6))
        sns.heatmap(cm, annot=True, fmt='d', cmap='Blues')
        plt.title(f'Matriz de Confusión - {model_name}')
        plt.xlabel('Predicción')
        plt.ylabel('Valor Real')
        
        if save_path:
            plt.savefig(save_path, dpi=300, bbox_inches='tight')
        
        plt.show()
\end{lstlisting}

\section{Sistema de Optimización de Hiperparámetros}

\subsection{Implementación de Grid Search y Random Search}

\begin{lstlisting}[language=Python, caption=Sistema de optimización de hiperparámetros]
from sklearn.model_selection import GridSearchCV, RandomizedSearchCV
from sklearn.model_selection import cross_val_score
import itertools
import numpy as np
import pandas as pd
from typing import Dict, Any, List
import time

class HyperparameterOptimizer:
    """Optimizador de hiperparámetros con múltiples estrategias"""
    
    def __init__(self, model_class, param_space: Dict[str, List], 
                 cv_folds: int = 5, scoring: str = 'accuracy'):
        self.model_class = model_class
        self.param_space = param_space
        self.cv_folds = cv_folds
        self.scoring = scoring
        self.results = {}
    
    def grid_search(self, X_train: pd.DataFrame, y_train: pd.Series, 
                   model_config: Dict[str, Any]) -> Dict[str, Any]:
        """Optimización por búsqueda en grilla"""
        
        print("Iniciando Grid Search...")
        start_time = time.time()
        
        # Crear modelo base
        base_model = self.model_class("grid_search", model_config)
        model = base_model.build_model()
        
        # Configurar Grid Search
        grid_search = GridSearchCV(
            estimator=model,
            param_grid=self.param_space,
            cv=self.cv_folds,
            scoring=self.scoring,
            n_jobs=-1,
            verbose=1
        )
        
        # Ejecutar búsqueda
        grid_search.fit(X_train, y_train)
        
        end_time = time.time()
        
        results = {
            'best_params': grid_search.best_params_,
            'best_score': grid_search.best_score_,
            'best_estimator': grid_search.best_estimator_,
            'cv_results': pd.DataFrame(grid_search.cv_results_),
            'search_time': end_time - start_time,
            'method': 'grid_search'
        }
        
        self.results['grid_search'] = results
        
        print(f"Grid Search completado en {results['search_time']:.2f} segundos")
        print(f"Mejor score: {results['best_score']:.4f}")
        print(f"Mejores parámetros: {results['best_params']}")
        
        return results
    
    def random_search(self, X_train: pd.DataFrame, y_train: pd.Series, 
                     model_config: Dict[str, Any], n_iter: int = 100) -> Dict[str, Any]:
        """Optimización por búsqueda aleatoria"""
        
        print("Iniciando Random Search...")
        start_time = time.time()
        
        # Crear modelo base
        base_model = self.model_class("random_search", model_config)
        model = base_model.build_model()
        
        # Configurar Random Search
        random_search = RandomizedSearchCV(
            estimator=model,
            param_distributions=self.param_space,
            n_iter=n_iter,
            cv=self.cv_folds,
            scoring=self.scoring,
            n_jobs=-1,
            verbose=1,
            random_state=42
        )
        
        # Ejecutar búsqueda
        random_search.fit(X_train, y_train)
        
        end_time = time.time()
        
        results = {
            'best_params': random_search.best_params_,
            'best_score': random_search.best_score_,
            'best_estimator': random_search.best_estimator_,
            'cv_results': pd.DataFrame(random_search.cv_results_),
            'search_time': end_time - start_time,
            'method': 'random_search',
            'n_iter': n_iter
        }
        
        self.results['random_search'] = results
        
        print(f"Random Search completado en {results['search_time']:.2f} segundos")
        print(f"Mejor score: {results['best_score']:.4f}")
        print(f"Mejores parámetros: {results['best_params']}")
        
        return results
    
    def bayesian_optimization(self, X_train: pd.DataFrame, y_train: pd.Series, 
                            model_config: Dict[str, Any], n_calls: int = 50):
        """Optimización bayesiana usando scikit-optimize"""
        
        try:
            from skopt import gp_minimize
            from skopt.space import Real, Integer, Categorical
            from skopt.utils import use_named_args
        except ImportError:
            print("scikit-optimize no está instalado. Usa: pip install scikit-optimize")
            return None
        
        print("Iniciando Optimización Bayesiana...")
        start_time = time.time()
        
        # Convertir espacio de parámetros al formato de skopt
        dimensions = []
        param_names = []
        
        for param_name, param_values in self.param_space.items():
            param_names.append(param_name)
            
            if isinstance(param_values[0], float):
                dimensions.append(Real(min(param_values), max(param_values), 
                                     name=param_name))
            elif isinstance(param_values[0], int):
                dimensions.append(Integer(min(param_values), max(param_values), 
                                        name=param_name))
            else:
                dimensions.append(Categorical(param_values, name=param_name))
        
        # Función objetivo
        @use_named_args(dimensions)
        def objective(**params):
            # Crear modelo con parámetros actuales
            current_config = model_config.copy()
            current_config['model_params'] = params
            
            model = self.model_class("bayesian_opt", current_config)
            estimator = model.build_model()
            
            # Evaluación por validación cruzada
            scores = cross_val_score(estimator, X_train, y_train, 
                                   cv=self.cv_folds, scoring=self.scoring)
            
            # Devolver el negativo del score (minimización)
            return -np.mean(scores)
        
        # Ejecutar optimización
        result = gp_minimize(objective, dimensions, n_calls=n_calls, 
                           random_state=42)
        
        end_time = time.time()
        
        # Crear diccionario de mejores parámetros
        best_params = {}
        for i, param_name in enumerate(param_names):
            best_params[param_name] = result.x[i]
        
        results = {
            'best_params': best_params,
            'best_score': -result.fun,  # Convertir de vuelta a positivo
            'optimization_result': result,
            'search_time': end_time - start_time,
            'method': 'bayesian_optimization',
            'n_calls': n_calls
        }
        
        self.results['bayesian_optimization'] = results
        
        print(f"Optimización Bayesiana completada en {results['search_time']:.2f} segundos")
        print(f"Mejor score: {results['best_score']:.4f}")
        print(f"Mejores parámetros: {results['best_params']}")
        
        return results
    
    def compare_optimization_methods(self) -> pd.DataFrame:
        """Compara los resultados de diferentes métodos de optimización"""
        
        if not self.results:
            print("No hay resultados de optimización para comparar")
            return pd.DataFrame()
        
        comparison_data = []
        
        for method, result in self.results.items():
            comparison_data.append({
                'Method': method,
                'Best Score': result['best_score'],
                'Search Time (s)': result['search_time'],
                'Best Params': str(result['best_params'])
            })
        
        comparison_df = pd.DataFrame(comparison_data)
        return comparison_df.sort_values('Best Score', ascending=False)
\end{lstlisting}

\section{API de Predicción}

\subsection{Implementación con FastAPI}

\begin{lstlisting}[language=Python, caption=API REST para predicciones en tiempo real]
from fastapi import FastAPI, HTTPException
from pydantic import BaseModel
import pandas as pd
import joblib
import numpy as np
from typing import List, Dict, Any
import logging

# Configurar logging
logging.basicConfig(level=logging.INFO)
logger = logging.getLogger(__name__)

# Inicializar FastAPI
app = FastAPI(
    title="Modelo de Analítica de Datos API",
    description="API para predicciones en tiempo real",
    version="1.0.0"
)

# Modelos cargados en memoria
loaded_models = {}
preprocessor = None

class PredictionRequest(BaseModel):
    """Schema para solicitud de predicción"""
    features: Dict[str, Any]
    model_name: str = "best_model"

class PredictionResponse(BaseModel):
    """Schema para respuesta de predicción"""
    prediction: float
    probability: List[float] = None
    model_used: str
    confidence: float = None

class BatchPredictionRequest(BaseModel):
    """Schema para predicciones en lote"""
    data: List[Dict[str, Any]]
    model_name: str = "best_model"

@app.on_event("startup")
async def load_models():
    """Carga modelos al iniciar la API"""
    global loaded_models, preprocessor
    
    try:
        # Cargar preprocesador
        preprocessor = joblib.load("models/preprocessor.pkl")
        logger.info("Preprocesador cargado exitosamente")
        
        # Cargar modelos entrenados
        model_files = [
            ("best_model", "models/best_model.pkl"),
            ("random_forest", "models/random_forest_model.pkl"),
            ("xgboost", "models/xgboost_model.pkl")
        ]
        
        for model_name, model_path in model_files:
            try:
                loaded_models[model_name] = joblib.load(model_path)
                logger.info(f"Modelo {model_name} cargado desde {model_path}")
            except FileNotFoundError:
                logger.warning(f"Archivo {model_path} no encontrado")
        
        if not loaded_models:
            logger.error("No se pudo cargar ningún modelo")
        
    except Exception as e:
        logger.error(f"Error al cargar modelos: {str(e)}")

@app.get("/")
async def root():
    """Endpoint raíz"""
    return {
        "message": "API de Analítica de Datos",
        "version": "1.0.0",
        "models_available": list(loaded_models.keys())
    }

@app.get("/health")
async def health_check():
    """Verificación de salud de la API"""
    return {
        "status": "healthy",
        "models_loaded": len(loaded_models),
        "preprocessor_loaded": preprocessor is not None
    }

@app.get("/models")
async def list_models():
    """Lista modelos disponibles"""
    model_info = {}
    
    for name, model in loaded_models.items():
        model_info[name] = {
            "type": type(model).__name__,
            "features": getattr(model, 'feature_names_in_', None)
        }
    
    return {"available_models": model_info}

@app.post("/predict", response_model=PredictionResponse)
async def predict(request: PredictionRequest):
    """Realizar predicción individual"""
    
    # Verificar que el modelo existe
    if request.model_name not in loaded_models:
        raise HTTPException(
            status_code=404, 
            detail=f"Modelo '{request.model_name}' no encontrado"
        )
    
    try:
        # Convertir features a DataFrame
        input_df = pd.DataFrame([request.features])
        
        # Aplicar preprocesamiento si está disponible
        if preprocessor is not None:
            input_processed = preprocessor.transform(input_df)
            if hasattr(input_processed, 'toarray'):  # Matriz dispersa
                input_processed = input_processed.toarray()
        else:
            input_processed = input_df.values
        
        # Obtener modelo
        model = loaded_models[request.model_name]
        
        # Realizar predicción
        prediction = model.predict(input_processed)[0]
        
        # Obtener probabilidades si es posible
        probabilities = None
        confidence = None
        
        if hasattr(model, 'predict_proba'):
            proba = model.predict_proba(input_processed)[0]
            probabilities = proba.tolist()
            confidence = float(np.max(proba))
        
        return PredictionResponse(
            prediction=float(prediction),
            probability=probabilities,
            model_used=request.model_name,
            confidence=confidence
        )
        
    except Exception as e:
        logger.error(f"Error en predicción: {str(e)}")
        raise HTTPException(status_code=500, detail=f"Error en predicción: {str(e)}")

@app.post("/predict_batch")
async def predict_batch(request: BatchPredictionRequest):
    """Realizar predicciones en lote"""
    
    if request.model_name not in loaded_models:
        raise HTTPException(
            status_code=404,
            detail=f"Modelo '{request.model_name}' no encontrado"
        )
    
    try:
        # Convertir datos a DataFrame
        input_df = pd.DataFrame(request.data)
        
        # Aplicar preprocesamiento
        if preprocessor is not None:
            input_processed = preprocessor.transform(input_df)
            if hasattr(input_processed, 'toarray'):
                input_processed = input_processed.toarray()
        else:
            input_processed = input_df.values
        
        # Obtener modelo y realizar predicciones
        model = loaded_models[request.model_name]
        predictions = model.predict(input_processed)
        
        # Obtener probabilidades si es posible
        probabilities = None
        if hasattr(model, 'predict_proba'):
            probabilities = model.predict_proba(input_processed).tolist()
        
        return {
            "predictions": predictions.tolist(),
            "probabilities": probabilities,
            "model_used": request.model_name,
            "n_predictions": len(predictions)
        }
        
    except Exception as e:
        logger.error(f"Error en predicción por lotes: {str(e)}")
        raise HTTPException(
            status_code=500, 
            detail=f"Error en predicción por lotes: {str(e)}"
        )

@app.post("/retrain")
async def retrain_model():
    """Endpoint para reentrenar modelos (simplificado)"""
    # Este endpoint requeriría implementación adicional
    # para manejar nuevos datos de entrenamiento
    return {
        "message": "Funcionalidad de reentrenamiento no implementada",
        "status": "not_implemented"
    }

if __name__ == "__main__":
    import uvicorn
    uvicorn.run(app, host="0.0.0.0", port=8000)
\end{lstlisting}

\section{Desafíos y Soluciones}

\subsection{Desafíos Técnicos Encontrados}

\subsubsection{Gestión de Memoria}

\textbf{Problema:} Procesamiento de datasets grandes que excedían la memoria RAM disponible.

\textbf{Solución implementada:}
\begin{itemize}
    \item Implementación de procesamiento por chunks
    \item Uso de Dask para computación distribuida
    \item Optimización de tipos de datos (downcasting)
    \item Implementación de lazy loading
\end{itemize}

\subsubsection{Escalabilidad del Entrenamiento}

\textbf{Problema:} Tiempo excesivo de entrenamiento para modelos complejos.

\textbf{Solución implementada:}
\begin{itemize}
    \item Paralelización con joblib
    \item Uso de early stopping
    \item Implementación de warm start para modelos iterativos
    \item Optimización de hiperparámetros con métodos eficientes
\end{itemize}

\subsubsection{Reproducibilidad}

\textbf{Problema:} Inconsistencias en resultados entre diferentes ejecuciones.

\textbf{Solución implementada:}
\begin{itemize}
    \item Fijación de semillas aleatorias en todas las librerías
    \item Containerización con Docker
    \item Versionado de datos y modelos
    \item Documentación detallada de versiones de dependencias
\end{itemize}

\subsection{Optimizaciones Implementadas}

\begin{table}[htbp]
\centering
\caption{Optimizaciones implementadas y su impacto}
\begin{tabular}{@{}p{4cm}p{5cm}p{4cm}@{}}
\toprule
\textbf{Optimización} & \textbf{Descripción} & \textbf{Mejora Lograda} \\
\midrule
Procesamiento paralelo & Uso de múltiples cores para entrenamiento & 3x más rápido \\
Feature selection & Eliminación de características irrelevantes & 40\% reducción tiempo \\
Early stopping & Parada temprana en entrenamiento & 60\% reducción tiempo \\
Batch processing & Procesamiento por lotes de datos & 80\% menos memoria \\
Model caching & Cache de modelos entrenados & 10x más rápido inferencia \\
\bottomrule
\end{tabular}
\label{tab:optimizaciones}
\end{table}

\section{Conclusiones del Capítulo}

La implementación del sistema de analítica de datos ha resultado en una solución robusta y escalable que cumple con los objetivos planteados. Los principales logros incluyen:

\begin{enumerate}
    \item \textbf{Arquitectura Modular:} Sistema flexible que permite agregar nuevos modelos y técnicas fácilmente
    \item \textbf{Pipeline Automatizado:} Proceso end-to-end desde datos en bruto hasta predicciones
    \item \textbf{API de Producción:} Interfaz lista para deployment en entornos productivos
    \item \textbf{Sistema de Evaluación:} Métricas comprehensivas para validación de modelos
    \item \textbf{Optimización de Rendimiento:} Soluciones eficientes para datasets grandes
\end{enumerate}

El sistema implementado proporciona una base sólida para el análisis de resultados que se presenta en el siguiente capítulo.


% Capítulo 6: Resultados y Análisis
% ========================================================================
% CAPÍTULO 6: RESULTADOS Y ANÁLISIS
% ========================================================================

\chapter{Resultados y Análisis}

\section{Introducción}

Este capítulo presenta los resultados obtenidos durante la experimentación y evaluación de los modelos implementados. Se incluye el análisis detallado de los datasets utilizados, el rendimiento de los diferentes algoritmos aplicados, la comparación entre métodos, y la interpretación de los hallazgos principales.

\section{Caracterización de los Datos}

\subsection{Análisis Exploratorio del Dataset Principal}

\subsubsection{Estadísticas Descriptivas}

El dataset principal contiene [número] instancias con [número] características. La Tabla \ref{tab:estadisticas_descriptivas} presenta las estadísticas descriptivas principales.

\begin{table}[htbp]
\centering
\caption{Estadísticas descriptivas del dataset principal}
\begin{tabular}{@{}p{3cm}p{2cm}p{2cm}p{2cm}p{2cm}p{2cm}@{}}
\toprule
\textbf{Variable} & \textbf{Media} & \textbf{Std} & \textbf{Min} & \textbf{Max} & \textbf{Missing (\%)} \\
\midrule
[Variable 1] & [valor] & [valor] & [valor] & [valor] & [valor] \\
[Variable 2] & [valor] & [valor] & [valor] & [valor] & [valor] \\
[Variable 3] & [valor] & [valor] & [valor] & [valor] & [valor] \\
[Variable objetivo] & [valor] & [valor] & [valor] & [valor] & [valor] \\
\bottomrule
\end{tabular}
\label{tab:estadisticas_descriptivas}
\end{table}

\subsubsection{Distribución de Variables}

El análisis de distribuciones reveló patrones importantes:

\begin{itemize}
    \item \textbf{Variable objetivo:} Distribución [tipo: normal/sesgada/uniforme] con [características específicas]
    \item \textbf{Variables numéricas:} [Número] variables siguen distribución aproximadamente normal, [número] presentan sesgo
    \item \textbf{Variables categóricas:} [Análisis de frecuencias y balance de clases]
\end{itemize}

\begin{figure}[htbp]
\centering
% \includegraphics[width=0.8\textwidth]{imagenes/distribucion_variables.png}
\caption{Distribución de las principales variables del dataset}
\label{fig:distribucion_variables}
\end{figure}

\subsubsection{Análisis de Correlaciones}

La matriz de correlación (Figura \ref{fig:matriz_correlacion}) muestra:

\begin{itemize}
    \item \textbf{Correlaciones altas:} Variables [especificar] presentan correlación > 0.8
    \item \textbf{Relación con variable objetivo:} Las variables más correlacionadas son [especificar]
    \item \textbf{Multicolinealidad:} Identificadas [número] pares de variables con alta correlación
\end{itemize}

\begin{figure}[htbp]
\centering
% \includegraphics[width=0.9\textwidth]{imagenes/matriz_correlacion.png}
\caption{Matriz de correlación entre variables}
\label{fig:matriz_correlacion}
\end{figure}

\subsection{Calidad de los Datos}

\subsubsection{Valores Faltantes}

El análisis de valores faltantes reveló:

\begin{itemize}
    \item \textbf{Total de valores faltantes:} [porcentaje]% del dataset
    \item \textbf{Variables más afectadas:} [listar variables] con [porcentajes]% faltantes
    \item \textbf{Patrones de missingness:} [MCAR/MAR/MNAR según análisis]
\end{itemize}

\subsubsection{Detección de Outliers}

Utilizando el método IQR, se identificaron:

\begin{itemize}
    \item \textbf{Outliers totales:} [número] instancias ([porcentaje]% del dataset)
    \item \textbf{Variables más afectadas:} [variables específicas]
    \item \textbf{Criterio de manejo:} [describir estrategia adoptada]
\end{itemize}

\section{Resultados de Preprocesamiento}

\subsection{Impacto de las Transformaciones}

\subsubsection{Normalización de Datos}

La aplicación de diferentes técnicas de normalización mostró los siguientes resultados:

\begin{table}[htbp]
\centering
\caption{Comparación de técnicas de normalización}
\begin{tabular}{@{}p{4cm}p{3cm}p{3cm}p{3cm}@{}}
\toprule
\textbf{Técnica} & \textbf{Tiempo (seg)} & \textbf{Memoria (MB)} & \textbf{Impacto en Modelo} \\
\midrule
Min-Max Scaling & [valor] & [valor] & [descripción] \\
Standard Scaling & [valor] & [valor] & [descripción] \\
Robust Scaling & [valor] & [valor] & [descripción] \\
Sin normalización & [valor] & [valor] & [descripción] \\
\bottomrule
\end{tabular}
\label{tab:normalizacion}
\end{table}

\subsubsection{Ingeniería de Características}

La creación de nuevas características generó:

\begin{itemize}
    \item \textbf{Características originales:} [número]
    \item \textbf{Características generadas:} [número]
    \item \textbf{Características finales (después de selección):} [número]
    \item \textbf{Mejora en rendimiento:} [porcentaje]% en métrica principal
\end{itemize}

\section{Resultados de Modelado}

\subsection{Modelos Baseline}

\subsubsection{Regresión Lineal/Logística}

Los resultados del modelo baseline se presentan en la Tabla \ref{tab:resultados_baseline}.

\begin{table}[htbp]
\centering
\caption{Resultados de modelos baseline}
\begin{tabular}{@{}p{3cm}p{2.5cm}p{2.5cm}p{2.5cm}p{2.5cm}@{}}
\toprule
\textbf{Modelo} & \textbf{Accuracy/R²} & \textbf{Precision/RMSE} & \textbf{Recall/MAE} & \textbf{F1/MAPE} \\
\midrule
Regresión Lineal & [valor] & [valor] & [valor] & [valor] \\
Regresión Logística & [valor] & [valor] & [valor] & [valor] \\
Naive Bayes & [valor] & [valor] & [valor] & [valor] \\
k-NN & [valor] & [valor] & [valor] & [valor] \\
\bottomrule
\end{tabular}
\label{tab:resultados_baseline}
\end{table}

\subsection{Modelos Avanzados}

\subsubsection{Random Forest}

\textbf{Configuración óptima encontrada:}
\begin{itemize}
    \item n\_estimators: [valor]
    \item max\_depth: [valor]
    \item min\_samples\_split: [valor]
    \item max\_features: [valor]
\end{itemize}

\textbf{Resultados:}
\begin{itemize}
    \item Accuracy/R²: [valor] (±[intervalo de confianza])
    \item Tiempo de entrenamiento: [valor] segundos
    \item Tiempo de predicción: [valor] ms por muestra
\end{itemize}

\subsubsection{XGBoost}

\textbf{Configuración óptima encontrada:}
\begin{itemize}
    \item learning\_rate: [valor]
    \item max\_depth: [valor]
    \item n\_estimators: [valor]
    \item subsample: [valor]
\end{itemize}

\textbf{Resultados:}
\begin{itemize}
    \item Accuracy/R²: [valor] (±[intervalo de confianza])
    \item Tiempo de entrenamiento: [valor] segundos
    \item Tiempo de predicción: [valor] ms por muestra
\end{itemize}

\subsubsection{Redes Neuronales}

\textbf{Arquitectura óptima:}
\begin{itemize}
    \item Capas ocultas: [configuración]
    \item Función de activación: [función]
    \item Optimizer: [optimizador]
    \item Learning rate: [valor]
\end{itemize}

\textbf{Curvas de aprendizaje:}

La Figura \ref{fig:curvas_aprendizaje} muestra la evolución del loss durante el entrenamiento.

\begin{figure}[htbp]
\centering
% \includegraphics[width=0.8\textwidth]{imagenes/curvas_aprendizaje.png}
\caption{Curvas de aprendizaje para red neuronal}
\label{fig:curvas_aprendizaje}
\end{figure}

\section{Comparación de Modelos}

\subsection{Rendimiento Global}

La Tabla \ref{tab:comparacion_modelos} presenta la comparación completa de todos los modelos evaluados.

\begin{table}[htbp]
\centering
\caption{Comparación de rendimiento entre modelos}
\begin{tabular}{@{}p{3cm}p{2cm}p{2cm}p{2cm}p{2cm}p{2cm}@{}}
\toprule
\textbf{Modelo} & \textbf{Accuracy} & \textbf{Precision} & \textbf{Recall} & \textbf{F1-Score} & \textbf{AUC-ROC} \\
\midrule
Regresión Logística & 0.756 & 0.742 & 0.751 & 0.746 & 0.823 \\
Random Forest & 0.834 & 0.829 & 0.831 & 0.830 & 0.891 \\
XGBoost & \textbf{0.847} & \textbf{0.842} & \textbf{0.845} & \textbf{0.844} & \textbf{0.903} \\
Red Neuronal & 0.829 & 0.825 & 0.828 & 0.826 & 0.887 \\
SVM & 0.798 & 0.795 & 0.797 & 0.796 & 0.856 \\
\bottomrule
\end{tabular}
\label{tab:comparacion_modelos}
\end{table}

\subsection{Análisis Estadístico}

\subsubsection{Pruebas de Significancia}

Se realizaron pruebas t de Student para comparar el rendimiento entre modelos:

\begin{itemize}
    \item \textbf{XGBoost vs Random Forest:} p-valor = [valor], diferencia [significativa/no significativa]
    \item \textbf{XGBoost vs Red Neuronal:} p-valor = [valor], diferencia [significativa/no significativa]
    \item \textbf{Random Forest vs SVM:} p-valor = [valor], diferencia [significativa/no significativa]
\end{itemize}

\subsubsection{Intervalos de Confianza}

Los intervalos de confianza del 95\% para la métrica principal son:

\begin{itemize}
    \item \textbf{XGBoost:} [valor inferior, valor superior]
    \item \textbf{Random Forest:} [valor inferior, valor superior]
    \item \textbf{Red Neuronal:} [valor inferior, valor superior]
\end{itemize}

\subsection{Análisis de Eficiencia}

\begin{figure}[htbp]
\centering
% \includegraphics[width=0.8\textwidth]{imagenes/rendimiento_tiempo.png}
\caption{Relación entre rendimiento y tiempo de entrenamiento}
\label{fig:rendimiento_tiempo}
\end{figure}

La Figura \ref{fig:rendimiento_tiempo} muestra el trade-off entre precisión y tiempo de entrenamiento:

\begin{itemize}
    \item \textbf{Mejor eficiencia:} Random Forest (alta precisión, tiempo moderado)
    \item \textbf{Mayor precisión:} XGBoost (mejor resultado, mayor tiempo)
    \item \textbf{Más rápido:} Regresión Logística (tiempo mínimo, precisión aceptable)
\end{itemize}

\section{Análisis de Características}

\subsection{Importancia de Variables}

\subsubsection{Random Forest - Feature Importance}

\begin{table}[htbp]
\centering
\caption{Top 10 características más importantes (Random Forest)}
\begin{tabular}{@{}p{4cm}p{3cm}p{6cm}@{}}
\toprule
\textbf{Característica} & \textbf{Importancia} & \textbf{Interpretación} \\
\midrule
[Variable 1] & [valor] & [Descripción del impacto] \\
[Variable 2] & [valor] & [Descripción del impacto] \\
[Variable 3] & [valor] & [Descripción del impacto] \\
[Variable 4] & [valor] & [Descripción del impacto] \\
[Variable 5] & [valor] & [Descripción del impacto] \\
\bottomrule
\end{tabular}
\label{tab:feature_importance}
\end{table}

\subsubsection{Análisis SHAP}

Se aplicó SHAP (SHapley Additive exPlanations) para interpretar las predicciones del modelo XGBoost:

\begin{figure}[htbp]
\centering
% \includegraphics[width=0.9\textwidth]{imagenes/shap_summary.png}
\caption{Resumen de valores SHAP para características principales}
\label{fig:shap_summary}
\end{figure}

\textbf{Hallazgos principales del análisis SHAP:}

\begin{itemize}
    \item \textbf{[Variable más importante]:} Impacto positivo cuando [condición], negativo cuando [condición]
    \item \textbf{[Variable 2]:} Relación no lineal, mayor impacto en [rango de valores]
    \item \textbf{Interacciones:} Variables [X] e [Y] muestran efectos sinérgicos
\end{itemize}

\section{Validación del Modelo}

\subsection{Validación Cruzada}

Se realizó validación cruzada k-fold (k=5) para evaluar la robustez de los modelos:

\begin{table}[htbp]
\centering
\caption{Resultados de validación cruzada}
\begin{tabular}{@{}p{3cm}p{2cm}p{2cm}p{2cm}p{2cm}@{}}
\toprule
\textbf{Modelo} & \textbf{Media} & \textbf{Std} & \textbf{Min} & \textbf{Max} \\
\midrule
XGBoost & [valor] & [valor] & [valor] & [valor] \\
Random Forest & [valor] & [valor] & [valor] & [valor] \\
Red Neuronal & [valor] & [valor] & [valor] & [valor] \\
\bottomrule
\end{tabular}
\label{tab:validacion_cruzada}
\end{table}

\subsection{Análisis de Residuales}

Para modelos de regresión, el análisis de residuales reveló:

\begin{itemize}
    \item \textbf{Normalidad:} Test de Shapiro-Wilk p-valor = [valor]
    \item \textbf{Homocedasticidad:} Test de Breusch-Pagan p-valor = [valor]
    \item \textbf{Independencia:} Test de Durbin-Watson estadístico = [valor]
\end{itemize}

\subsection{Matriz de Confusión}

Para el mejor modelo de clasificación (XGBoost):

\begin{figure}[htbp]
\centering
% \includegraphics[width=0.6\textwidth]{imagenes/confusion_matrix.png}
\caption{Matriz de confusión del modelo XGBoost}
\label{fig:confusion_matrix}
\end{figure}

\textbf{Análisis de errores:}
\begin{itemize}
    \item \textbf{Falsos positivos:} [número] casos, principalmente en [categorías específicas]
    \item \textbf{Falsos negativos:} [número] casos, asociados con [características específicas]
    \item \textbf{Precisión por clase:} Clase [A]: [valor], Clase [B]: [valor]
\end{itemize}

\section{Optimización de Hiperparámetros}

\subsection{Comparación de Métodos}

\begin{table}[htbp]
\centering
\caption{Comparación de métodos de optimización de hiperparámetros}
\begin{tabular}{@{}p{3cm}p{2.5cm}p{2.5cm}p{3cm}p{3cm}@{}}
\toprule
\textbf{Método} & \textbf{Mejor Score} & \textbf{Tiempo (min)} & \textbf{Evaluaciones} & \textbf{Eficiencia} \\
\midrule
Grid Search & [valor] & [valor] & [valor] & [valor] \\
Random Search & [valor] & [valor] & [valor] & [valor] \\
Bayesian Opt. & [valor] & [valor] & [valor] & [valor] \\
\bottomrule
\end{tabular}
\label{tab:optimizacion_metodos}
\end{table}

\subsection{Convergencia de la Optimización}

\begin{figure}[htbp]
\centering
% \includegraphics[width=0.8\textwidth]{imagenes/convergencia_optimizacion.png}
\caption{Convergencia de la optimización bayesiana}
\label{fig:convergencia_optimizacion}
\end{figure}

La optimización bayesiana mostró:
\begin{itemize}
    \item \textbf{Convergencia:} Después de [número] evaluaciones
    \item \textbf{Mejor resultado:} Score de [valor] en evaluación [número]
    \item \textbf{Eficiencia:} [porcentaje]% mejor que búsqueda aleatoria
\end{itemize}

\section{Análisis de Casos Específicos}

\subsection{Casos de Éxito}

\textbf{Ejemplo 1:} [Descripción de caso específico donde el modelo funcionó excepcionalmente bien]

\begin{itemize}
    \item \textbf{Características:} [Describir las características específicas]
    \item \textbf{Predicción:} [Valor predicho vs valor real]
    \item \textbf{Confianza:} [Nivel de confianza del modelo]
\end{itemize}

\subsection{Casos de Fallo}

\textbf{Ejemplo 1:} [Descripción de caso donde el modelo falló]

\begin{itemize}
    \item \textbf{Razón del fallo:} [Análisis de por qué falló]
    \item \textbf{Características atípicas:} [Valores fuera de lo común]
    \item \textbf{Lecciones aprendidas:} [Cómo mejorar el modelo]
\end{itemize}

\section{Validación en Datos Reales}

\subsection{Implementación Piloto}

Se implementó el modelo en un entorno de prueba con datos reales durante [período]:

\begin{itemize}
    \item \textbf{Volumen de datos:} [número] predicciones realizadas
    \item \textbf{Precisión en producción:} [valor] (vs [valor] en test)
    \item \textbf{Tiempo de respuesta:} [valor] ms promedio
    \item \textbf{Disponibilidad del sistema:} [porcentaje]%
\end{itemize}

\subsection{Feedback de Usuarios}

El feedback de los usuarios finales reveló:

\begin{itemize}
    \item \textbf{Satisfacción general:} [escala/puntuación]
    \item \textbf{Utilidad de las predicciones:} [porcentaje] consideran útiles
    \item \textbf{Confianza en el sistema:} [nivel de confianza reportado]
    \item \textbf{Sugerencias de mejora:} [principales sugerencias]
\end{itemize}

\section{Impacto y Beneficios}

\subsection{Métricas de Negocio}

La implementación del modelo resultó en:

\begin{itemize}
    \item \textbf{Reducción de tiempo:} [porcentaje]% en [proceso específico]
    \item \textbf{Mejora en precisión:} [porcentaje]% comparado con método anterior
    \item \textbf{Ahorro de costos:} [cantidad] en [período específico]
    \item \textbf{Incremento en eficiencia:} [métrica específica]
\end{itemize}

\subsection{Valor Agregado}

\begin{itemize}
    \item \textbf{Automatización:} Proceso que antes requería [tiempo] ahora es automático
    \item \textbf{Escalabilidad:} Capacidad de procesar [volumen] datos sin intervención manual
    \item \textbf{Consistencia:} Eliminación de variabilidad humana en decisiones
    \item \textbf{Insights:} Nuevos patrones descubiertos en los datos
\end{itemize}

\section{Limitaciones Identificadas}

\subsection{Limitaciones del Modelo}

\begin{enumerate}
    \item \textbf{Sesgo en datos:} [Descripción del sesgo identificado]
    \item \textbf{Generalización:} [Limitaciones para generalizar a nuevos contextos]
    \item \textbf{Interpretabilidad:} [Dificultades para explicar decisiones complejas]
    \item \textbf{Drift de datos:} [Sensibilidad a cambios en la distribución de datos]
\end{enumerate}

\subsection{Limitaciones Técnicas}

\begin{enumerate}
    \item \textbf{Recursos computacionales:} [Requerimientos de hardware/software]
    \item \textbf{Latencia:} [Tiempo de respuesta en casos complejos]
    \item \textbf{Escalabilidad:} [Límites de volumen de datos procesables]
    \item \textbf{Mantenimiento:} [Frecuencia de reentrenamiento requerida]
\end{enumerate}

\section{Comparación con Estado del Arte}

\subsection{Benchmarking}

Comparación con trabajos previos en el mismo dominio:

\begin{table}[htbp]
\centering
\caption{Comparación con estado del arte}
\begin{tabular}{@{}p{4cm}p{2.5cm}p{2.5cm}p{4cm}@{}}
\toprule
\textbf{Trabajo} & \textbf{Métrica} & \textbf{Resultado} & \textbf{Observaciones} \\
\midrule
\cite{autor2022} & Accuracy & [valor] & [Comentarios sobre diferencias] \\
\cite{autor2023} & F1-Score & [valor] & [Comentarios sobre diferencias] \\
Nuestro trabajo & Accuracy & [valor] & [Descripción de mejoras] \\
Nuestro trabajo & F1-Score & [valor] & [Descripción de mejoras] \\
\bottomrule
\end{tabular}
\label{tab:comparacion_estado_arte}
\end{table}

\subsection{Contribuciones Específicas}

Nuestro trabajo presenta las siguientes mejoras respecto al estado del arte:

\begin{enumerate}
    \item \textbf{Mejora en precisión:} [porcentaje]% superior al mejor trabajo previo
    \item \textbf{Reducción en tiempo:} [porcentaje]% más rápido que métodos anteriores
    \item \textbf{Nuevo enfoque metodológico:} [Descripción de la innovación]
    \item \textbf{Aplicabilidad ampliada:} [Extensión a nuevos casos de uso]
\end{enumerate}

\section{Reproducibilidad}

\subsection{Verificación de Resultados}

Para asegurar la reproducibilidad:

\begin{itemize}
    \item \textbf{Semillas fijas:} Todos los experimentos con semilla aleatoria = 42
    \item \textbf{Versiones documentadas:} [Listar versiones de librerías clave]
    \item \textbf{Código disponible:} [URL del repositorio si aplica]
    \item \textbf{Datasets:} [Disponibilidad y acceso a los datos]
\end{itemize}

\subsection{Réplica de Experimentos}

Se realizaron [número] réplicas independientes de los experimentos principales:

\begin{itemize}
    \item \textbf{Variabilidad:} Desviación estándar de [valor] en métrica principal
    \item \textbf{Consistencia:} [porcentaje]% de experimentos dentro de ±[valor] del resultado medio
    \item \textbf{Robustez:} Resultados estables ante pequeñas variaciones en datos
\end{itemize}

\section{Conclusiones del Capítulo}

Los resultados obtenidos demuestran que:

\begin{enumerate}
    \item \textbf{Objetivos alcanzados:} Se logró [objetivo principal] con una mejora del [porcentaje]% respecto al baseline
    
    \item \textbf{Mejor modelo:} XGBoost mostró el mejor rendimiento general con [métricas específicas]
    
    \item \textbf{Características clave:} Las variables [especificar] son las más predictivas del modelo
    
    \item \textbf{Aplicabilidad práctica:} El sistema es viable para implementación en producción
    
    \item \textbf{Contribución al conocimiento:} Los hallazgos aportan nuevos insights sobre [dominio específico]
\end{enumerate}

Los resultados validan las hipótesis planteadas y confirman la efectividad del enfoque metodológico propuesto. El siguiente capítulo presenta las conclusiones finales y direcciones para trabajo futuro.


% Capítulo 7: Conclusiones y Trabajo Futuro
% ========================================================================
% CAPÍTULO 7: CONCLUSIONES Y TRABAJO FUTURO
% ========================================================================

\chapter{Conclusiones y Trabajo Futuro}

\section{Introducción}

Este capítulo final presenta las conclusiones principales derivadas de la investigación, resume las contribuciones realizadas al campo de la analítica de datos, analiza las limitaciones del trabajo, y propone direcciones para futuras investigaciones. Se evalúa el cumplimiento de los objetivos planteados y se reflexiona sobre el impacto y las implicaciones de los resultados obtenidos.

\section{Síntesis de la Investigación}

\subsection{Recapitulación del Problema}

Esta tesis abordó el problema de [descripción específica del problema], motivado por [contexto y justificación]. El objetivo principal fue [objetivo general], el cual se desglosó en [número] objetivos específicos que guiaron el desarrollo de la investigación.

\subsection{Enfoque Metodológico}

La metodología adoptada combinó [enfoques principales utilizados], incluyendo:

\begin{itemize}
    \item Análisis exhaustivo de datos mediante técnicas de analítica exploratoria
    \item Implementación de múltiples algoritmos de machine learning
    \item Evaluación comparativa rigurosa con métricas estándar
    \item Validación en entornos controlados y reales
\end{itemize}

El enfoque metodológico demostró ser robusto y apropiado para abordar la complejidad del problema planteado.

\section{Principales Hallazgos}

\subsection{Resultados Clave}

Los principales hallazgos de esta investigación incluyen:

\begin{enumerate}
    \item \textbf{Rendimiento del Modelo:} El algoritmo XGBoost alcanzó el mejor rendimiento con una precisión de [valor]%, superando al baseline en [porcentaje de mejora]%.
    
    \item \textbf{Características Predictivas:} Las variables [enumerar variables más importantes] mostraron mayor poder predictivo, explicando [porcentaje]% de la varianza en el modelo.
    
    \item \textbf{Eficiencia Computacional:} La solución implementada procesa [volumen] datos en [tiempo], cumpliendo con los requerimientos de tiempo real.
    
    \item \textbf{Generalización:} Los modelos desarrollados mantuvieron su rendimiento en diferentes subconjuntos de datos, demostrando robustez y capacidad de generalización.
    
    \item \textbf{Interpretabilidad:} El análisis SHAP reveló patrones interpretables que proporcionan insights valiosos sobre [dominio específico].
\end{enumerate}

\subsection{Validación de Hipótesis}

\subsubsection{Hipótesis Principal}

\textbf{Hipótesis:} [Enunciar hipótesis principal]

\textbf{Resultado:} La hipótesis principal fue \textbf{validada}. Los resultados muestran que [descripción específica de cómo se validó], con una mejora estadísticamente significativa (p-valor < 0.05) del [porcentaje]% en [métrica específica].

\subsubsection{Hipótesis Secundarias}

\begin{enumerate}
    \item \textbf{Hipótesis 1:} [Enunciar] - \textbf{[Validada/Rechazada]} - [Breve justificación]
    \item \textbf{Hipótesis 2:} [Enunciar] - \textbf{[Validada/Rechazada]} - [Breve justificación]
    \item \textbf{Hipótesis 3:} [Enunciar] - \textbf{[Validada/Rechazada]} - [Breve justificación]
\end{enumerate}

\section{Cumplimiento de Objetivos}

\subsection{Objetivo General}

\textbf{Objetivo:} [Enunciar objetivo general]

\textbf{Cumplimiento:} El objetivo general fue \textbf{alcanzado completamente}. Se desarrolló exitosamente [descripción de lo logrado], que demostró [resultados específicos]. El sistema implementado cumple con todos los criterios establecidos y supera las expectativas iniciales en términos de [aspectos específicos].

\subsection{Objetivos Específicos}

\begin{enumerate}
    \item \textbf{Objetivo 1:} [Enunciar objetivo específico 1]
        \begin{itemize}
            \item \textbf{Estado:} ✓ Completado
            \item \textbf{Evidencia:} [Descripción de cómo se cumplió]
            \item \textbf{Resultados:} [Resultados específicos obtenidos]
        \end{itemize}
    
    \item \textbf{Objetivo 2:} [Enunciar objetivo específico 2]
        \begin{itemize}
            \item \textbf{Estado:} ✓ Completado
            \item \textbf{Evidencia:} [Descripción de cómo se cumplió]
            \item \textbf{Resultados:} [Resultados específicos obtenidos]
        \end{itemize}
    
    \item \textbf{Objetivo 3:} [Enunciar objetivo específico 3]
        \begin{itemize}
            \item \textbf{Estado:} ✓ Completado
            \item \textbf{Evidencia:} [Descripción de cómo se cumplió]
            \item \textbf{Resultados:} [Resultados específicos obtenidos]
        \end{itemize}
    
    \item \textbf{Objetivo 4:} [Enunciar objetivo específico 4]
        \begin{itemize}
            \item \textbf{Estado:} ✓ Completado
            \item \textbf{Evidencia:} [Descripción de cómo se cumplió]
            \item \textbf{Resultados:} [Resultados específicos obtenidos]
        \end{itemize}
    
    \item \textbf{Objetivo 5:} [Enunciar objetivo específico 5]
        \begin{itemize}
            \item \textbf{Estado:} ✓ Completado
            \item \textbf{Evidencia:} [Descripción de cómo se cumplió]
            \item \textbf{Resultados:} [Resultados específicos obtenidos]
        \end{itemize}
\end{enumerate}

\section{Contribuciones de la Investigación}

\subsection{Contribuciones Teóricas}

\begin{enumerate}
    \item \textbf{Marco Metodológico:} Se propuso un framework integrado que combina [técnicas específicas] para abordar [problema específico]. Este marco puede ser aplicado a problemas similares en [dominio].
    
    \item \textbf{Nuevos Insights:} El análisis reveló patrones previamente no documentados sobre [aspectos específicos], contribuyendo al entendimiento teórico de [área de conocimiento].
    
    \item \textbf{Extensión de Técnicas Existentes:} Se adaptaron y mejoraron técnicas tradicionales de [área específica] para el contexto de [aplicación específica].
\end{enumerate}

\subsection{Contribuciones Metodológicas}

\begin{enumerate}
    \item \textbf{Pipeline de Preprocesamiento:} Se desarrolló un pipeline automatizado que mejora la calidad de datos en [porcentaje]% comparado con métodos tradicionales.
    
    \item \textbf{Optimización de Hiperparámetros:} Se implementó una estrategia híbrida que reduce el tiempo de optimización en [porcentaje]% manteniendo la calidad de resultados.
    
    \item \textbf{Sistema de Evaluación:} Se creó un framework comprehensivo de evaluación que incluye métricas específicas para [dominio de aplicación].
\end{enumerate}

\subsection{Contribuciones Técnicas}

\begin{enumerate}
    \item \textbf{Implementación Escalable:} Se desarrolló una arquitectura que soporta [volumen específico] de datos con tiempos de respuesta menores a [tiempo específico].
    
    \item \textbf{API de Producción:} Se creó una interfaz RESTful que permite la integración fácil con sistemas existentes.
    
    \item \textbf{Herramientas de Monitoreo:} Se implementaron mecanismos de monitoreo de performance y drift de datos en tiempo real.
\end{enumerate}

\subsection{Contribuciones Prácticas}

\begin{enumerate}
    \item \textbf{Aplicación Real:} El sistema fue validado en un entorno de producción, demostrando su viabilidad práctica.
    
    \item \textbf{Transferencia de Conocimiento:} Se generó documentación detallada y código reutilizable para facilitar la adopción por otros investigadores y profesionales.
    
    \item \textbf{Impacto en el Dominio:} Los resultados proporcionan herramientas prácticas para [beneficiarios específicos] en [sector específico].
\end{enumerate}

\section{Impacto y Relevancia}

\subsection{Impacto Académico}

\begin{itemize}
    \item \textbf{Publicaciones:} Los resultados han sido/serán presentados en [conferencias/revistas específicas]
    \item \textbf{Citaciones:} El trabajo contribuye al corpus de conocimiento en [área específica]
    \item \textbf{Colaboraciones:} Ha generado colaboraciones con [instituciones/investigadores]
    \item \textbf{Tesis futuras:} Proporciona base para investigaciones posteriores
\end{itemize}

\subsection{Impacto Industrial}

\begin{itemize}
    \item \textbf{Adopción:} [Número] organizaciones han expresado interés en implementar la solución
    \item \textbf{Mejoras operacionales:} Potencial de reducir costos en [porcentaje]% en [proceso específico]
    \item \textbf{Nuevos productos:} Base para desarrollo de productos comerciales
    \item \textbf{Estándares:} Contribuye al establecimiento de mejores prácticas en [área]
\end{itemize}

\subsection{Impacto Social}

\begin{itemize}
    \item \textbf{Beneficios directos:} [Descripción de beneficios para la sociedad]
    \item \textbf{Accesibilidad:} Democratización de técnicas avanzadas de analítica
    \item \textbf{Transparencia:} Mejora en la interpretabilidad de sistemas de decisión
    \item \textbf{Ética:} Consideraciones de fairness y sesgo algorítmico
\end{itemize}

\section{Limitaciones del Estudio}

\subsection{Limitaciones Metodológicas}

\begin{enumerate}
    \item \textbf{Selección de Datos:} El estudio se limitó a datos de [fuente específica], lo que puede afectar la generalización a otros contextos.
    
    \item \textbf{Periodo Temporal:} Los datos analizados corresponden al periodo [período específico], por lo que los resultados pueden no capturar variaciones estacionales a largo plazo.
    
    \item \textbf{Métricas de Evaluación:} Aunque se utilizaron métricas estándar, podrían existir métricas específicas del dominio más apropiadas.
    
    \item \textbf{Comparación con Estado del Arte:} La comparación se limitó a trabajos disponibles públicamente, excluyendo potenciales soluciones propietarias.
\end{enumerate}

\subsection{Limitaciones Técnicas}

\begin{enumerate}
    \item \textbf{Recursos Computacionales:} Los experimentos se realizaron con recursos limitados, lo que restringió la exploración de arquitecturas más complejas.
    
    \item \textbf{Escalabilidad:} Aunque el sistema es escalable, no se probó con volúmenes de datos superiores a [volumen específico].
    
    \item \textbf{Latencia:} Los tiempos de respuesta, aunque aceptables, podrían ser críticos en aplicaciones de tiempo real estricto.
    
    \item \textbf{Dependencias:} El sistema depende de librerías específicas que pueden cambiar en versiones futuras.
\end{enumerate}

\subsection{Limitaciones de Datos}

\begin{enumerate}
    \item \textbf{Calidad de Datos:} Presencia de ruido y valores faltantes que, aunque tratados, pueden afectar los resultados.
    
    \item \textbf{Sesgo de Selección:} Los datos pueden no ser completamente representativos de la población objetivo.
    
    \item \textbf{Privacidad:} Restricciones de privacidad limitaron el acceso a ciertos tipos de datos potencialmente útiles.
    
    \item \textbf{Etiquetado:} La calidad del etiquetado manual puede introducir inconsistencias.
\end{enumerate}

\section{Lecciones Aprendidas}

\subsection{Aspectos Técnicos}

\begin{enumerate}
    \item \textbf{Preprocesamiento es Clave:} La calidad del preprocesamiento de datos tuvo mayor impacto en el rendimiento final que la complejidad del algoritmo.
    
    \item \textbf{Importancia de la Validación:} La validación cruzada y en datos reales reveló diferencias significativas respecto a las métricas de entrenamiento.
    
    \item \textbf{Trade-offs de Rendimiento:} Existe un balance crítico entre precisión, velocidad e interpretabilidad que debe considerarse según el contexto de aplicación.
    
    \item \textbf{Monitoreo Continuo:} Los modelos requieren monitoreo constante para detectar degradación de rendimiento por drift de datos.
\end{enumerate}

\subsection{Aspectos Metodológicos}

\begin{enumerate}
    \item \textbf{Iteración Rápida:} Un enfoque iterativo e incremental permitió identificar y corregir problemas tempranamente.
    
    \item \textbf{Colaboración Interdisciplinaria:} La colaboración con expertos del dominio fue fundamental para la interpretación correcta de resultados.
    
    \item \textbf{Documentación:} La documentación detallada desde el inicio facilitó la reproducibilidad y colaboración.
    
    \item \textbf{Gestión de Expectativas:} La comunicación clara de limitaciones y supuestos previno expectativas irreales.
\end{enumerate}

\subsection{Aspectos de Gestión}

\begin{enumerate}
    \item \textbf{Planificación de Recursos:} La estimación inicial de recursos computacionales fue insuficiente para algunos experimentos.
    
    \item \textbf{Gestión de Versiones:} Un sistema robusto de control de versiones para datos y modelos es esencial.
    
    \item \textbf{Backup y Recuperación:} Estrategias de backup previenen pérdida de trabajo por fallas técnicas.
    
    \item \textbf{Comunicación de Resultados:} La visualización efectiva de resultados es tan importante como los resultados mismos.
\end{enumerate}

\section{Trabajo Futuro}

\subsection{Mejoras a Corto Plazo}

\subsubsection{Optimizaciones Técnicas}

\begin{enumerate}
    \item \textbf{Paralelización Avanzada:} Implementar procesamiento distribuido para manejar datasets de mayor escala.
    
    \item \textbf{Optimización de Memoria:} Desarrollar técnicas de streaming para reducir el uso de memoria.
    
    \item \textbf{Aceleración por Hardware:} Explorar el uso de GPUs y TPUs para acelerar el entrenamiento.
    
    \item \textbf{Compresión de Modelos:} Investigar técnicas de pruning y quantización para modelos más ligeros.
\end{enumerate}

\subsubsection{Mejoras en Algoritmos}

\begin{enumerate}
    \item \textbf{Ensemble Avanzados:} Explorar técnicas de stacking y blending más sofisticadas.
    
    \item \textbf{Transfer Learning:} Investigar la aplicación de modelos pre-entrenados.
    
    \item \textbf{AutoML:} Implementar técnicas de aprendizaje automático de arquitecturas.
    
    \item \textbf{Aprendizaje Continuo:} Desarrollar capacidades de aprendizaje incremental.
\end{enumerate}

\subsection{Extensiones a Mediano Plazo}

\subsubsection{Nuevos Dominios de Aplicación}

\begin{enumerate}
    \item \textbf{Sector Salud:} Adaptar la metodología para diagnóstico médico asistido.
    
    \item \textbf{Finanzas:} Aplicar técnicas desarrolladas para detección de fraude y análisis de riesgo.
    
    \item \textbf{IoT:} Extender a análisis de datos de sensores y dispositivos conectados.
    
    \item \textbf{Smart Cities:} Aplicar en optimización de tráfico y gestión urbana.
\end{enumerate}

\subsubsection{Investigación Avanzada}

\begin{enumerate}
    \item \textbf{Explicabilidad:} Desarrollar técnicas más avanzadas de interpretabilidad de modelos complejos.
    
    \item \textbf{Fairness:} Investigar métodos para garantizar equidad algorítmica.
    
    \item \textbf{Robustez:} Estudiar la resistencia a ataques adversariales.
    
    \item \textbf{Causalidad:} Incorporar análisis causal para mejor entendimiento de relaciones.
\end{enumerate}

\subsection{Visión a Largo Plazo}

\subsubsection{Investigación Fundamental}

\begin{enumerate}
    \item \textbf{Nueva Teoría:} Contribuir al desarrollo de fundamentos teóricos en [área específica].
    
    \item \textbf{Interdisciplinariedad:} Establecer puentes entre analítica de datos y [disciplinas específicas].
    
    \item \textbf{Metodologías Híbridas:} Desarrollar enfoques que combinen múltiples paradigmas.
    
    \item \textbf{Estándares:} Contribuir al desarrollo de estándares de la industria.
\end{enumerate}

\subsubsection{Impacto Transformacional}

\begin{enumerate}
    \item \textbf{Democratización:} Hacer accesibles técnicas avanzadas a organizaciones sin recursos especializados.
    
    \item \textbf{Automatización Inteligente:} Desarrollar sistemas que tomen decisiones complejas con mínima intervención humana.
    
    \item \textbf{Sostenibilidad:} Aplicar analítica para abordar desafíos de sostenibilidad ambiental.
    
    \item \textbf{Educación:} Desarrollar herramientas educativas para formar nueva generación de analistas de datos.
\end{enumerate}

\section{Recomendaciones}

\subsection{Para Investigadores}

\begin{enumerate}
    \item \textbf{Reproducibilidad:} Asegurar que todos los experimentos sean completamente reproducibles mediante documentación detallada y código versionado.
    
    \item \textbf{Validación Robusta:} Implementar múltiples formas de validación incluyendo datos externos al conjunto original.
    
    \item \textbf{Consideraciones Éticas:} Incluir análisis de bias, fairness y explicabilidad desde las primeras etapas.
    
    \item \textbf{Colaboración:} Fomentar colaboraciones interdisciplinarias para abordar problemas complejos.
\end{enumerate}

\subsection{Para Profesionales}

\begin{enumerate}
    \item \textbf{Incrementalidad:} Adoptar un enfoque incremental para la implementación de soluciones complejas.
    
    \item \textbf{Monitoreo:} Establecer sistemas robustos de monitoreo para detectar degradación de modelos.
    
    \item \textbf{Capacitación:} Invertir en capacitación continua para mantenerse actualizado con nuevas técnicas.
    
    \item \textbf{Documentación:} Mantener documentación detallada de decisiones y procesos para facilitar mantenimiento.
\end{enumerate}

\subsection{Para Organizaciones}

\begin{enumerate}
    \item \textbf{Infraestructura:} Invertir en infraestructura robusta que soporte escalabilidad futura.
    
    \item \textbf{Gobernanza:} Establecer marcos de gobernanza claros para el uso de analítica de datos.
    
    \item \textbf{Cultura de Datos:} Fomentar una cultura organizacional que valore la toma de decisiones basada en datos.
    
    \item \textbf{Ética:} Implementar políticas claras sobre el uso ético de datos y algoritmos.
\end{enumerate}

\section{Reflexiones Finales}

\subsection{Contribución al Conocimiento}

Esta investigación contribuye significativamente al campo de la analítica de datos mediante [resumen de contribuciones principales]. Los resultados obtenidos no solo validan la efectividad del enfoque propuesto, sino que también abren nuevas líneas de investigación que pueden beneficiar tanto a la academia como a la industria.

\subsection{Impacto Personal y Profesional}

El desarrollo de esta tesis ha proporcionado una experiencia de aprendizaje transformadora, permitiendo:

\begin{itemize}
    \item Dominio de técnicas avanzadas de analítica de datos
    \item Desarrollo de habilidades de investigación rigurosa
    \item Comprensión profunda de los desafíos prácticos en la implementación de soluciones
    \item Capacidad para comunicar resultados técnicos complejos
\end{itemize}

\subsection{Perspectiva de la Disciplina}

La analítica de datos continúa evolucionando rápidamente, con nuevas técnicas y aplicaciones emergiendo constantemente. Esta investigación se posiciona en la frontera del conocimiento actual y proporciona una base sólida para futuras innovaciones en el campo.

El futuro de la analítica de datos promete ser aún más emocionante, con el potencial de transformar prácticamente todos los aspectos de la sociedad moderna. Los profesionales e investigadores en este campo tienen la responsabilidad de asegurar que estos avances se utilicen de manera ética y beneficiosa para la humanidad.

\section{Conclusión Final}

Esta tesis ha demostrado exitosamente que [conclusión principal resumida]. Los objetivos planteados fueron alcanzados, las hipótesis validadas, y se realizaron contribuciones significativas al campo de la analítica de datos.

Los resultados obtenidos proporcionan evidencia sólida de que [afirmación principal basada en resultados], lo cual tiene implicaciones importantes para [área de aplicación]. La metodología desarrollada puede ser aplicada a problemas similares, y el sistema implementado está listo para su adopción en entornos de producción.

Mirando hacia el futuro, esta investigación establece una base sólida para futuras innovaciones en [área específica] y abre múltiples oportunidades de investigación que pueden continuar ampliando las fronteras del conocimiento en analítica de datos.

La experiencia de desarrollar esta tesis ha reforzado la convicción de que la analítica de datos, cuando se aplica de manera rigurosa y ética, tiene el potencial de generar impactos positivos significativos en la sociedad. Es responsabilidad de los investigadores y profesionales en este campo continuar explorando estas posibilidades mientras mantienen los más altos estándares de integridad científica y consideración ética.


% ========================================================================
% MATERIAL ADICIONAL
% ========================================================================

\backmatter

% Bibliografía
\printbibliography[title=Referencias Bibliográficas]

% Apéndices
\appendix
% ========================================================================
% APÉNDICE A: CÓDIGO FUENTE PRINCIPAL
% ========================================================================

\chapter{Código Fuente Principal}
\label{apendice:codigo}

\section{Introducción}

Este apéndice contiene el código fuente principal desarrollado durante la investigación. Se incluyen los módulos más importantes del sistema implementado, organizados por funcionalidad.

\section{Módulo de Preprocesamiento de Datos}

\subsection{Clase Principal de Preprocesamiento}

\begin{lstlisting}[language=Python, caption=Módulo principal de preprocesamiento]
import pandas as pd
import numpy as np
from sklearn.preprocessing import StandardScaler, LabelEncoder
from sklearn.impute import SimpleImputer
from sklearn.model_selection import train_test_split
import logging

class DataPreprocessor:
    """
    Clase principal para preprocesamiento de datos
    Maneja limpieza, transformación y división de datos
    """
    
    def __init__(self, config=None):
        self.config = config or {}
        self.scalers = {}
        self.encoders = {}
        self.imputers = {}
        self.feature_names = []
        self.logger = logging.getLogger(__name__)
        
    def load_data(self, file_path):
        """Carga datos desde archivo CSV"""
        try:
            data = pd.read_csv(file_path)
            self.logger.info(f"Datos cargados: {data.shape}")
            return data
        except Exception as e:
            self.logger.error(f"Error cargando datos: {e}")
            raise
    
    def clean_data(self, df):
        """Limpieza básica de datos"""
        initial_shape = df.shape
        
        # Eliminar duplicados
        df = df.drop_duplicates()
        
        # Eliminar columnas con >90% valores faltantes
        threshold = 0.9
        df = df.dropna(thresh=int(threshold * len(df)), axis=1)
        
        final_shape = df.shape
        self.logger.info(f"Limpieza completada: {initial_shape} -> {final_shape}")
        
        return df
    
    def handle_missing_values(self, df, strategy='mean'):
        """Manejo de valores faltantes"""
        numeric_columns = df.select_dtypes(include=[np.number]).columns
        categorical_columns = df.select_dtypes(include=['object']).columns
        
        # Imputar valores numéricos
        if len(numeric_columns) > 0:
            if 'numeric_imputer' not in self.imputers:
                self.imputers['numeric_imputer'] = SimpleImputer(strategy=strategy)
                df[numeric_columns] = self.imputers['numeric_imputer'].fit_transform(
                    df[numeric_columns]
                )
            else:
                df[numeric_columns] = self.imputers['numeric_imputer'].transform(
                    df[numeric_columns]
                )
        
        # Imputar valores categóricos
        if len(categorical_columns) > 0:
            if 'categorical_imputer' not in self.imputers:
                self.imputers['categorical_imputer'] = SimpleImputer(
                    strategy='most_frequent'
                )
                df[categorical_columns] = self.imputers['categorical_imputer'].fit_transform(
                    df[categorical_columns]
                )
            else:
                df[categorical_columns] = self.imputers['categorical_imputer'].transform(
                    df[categorical_columns]
                )
        
        return df
    
    def encode_categorical_variables(self, df, target_column=None):
        """Codificación de variables categóricas"""
        categorical_columns = df.select_dtypes(include=['object']).columns
        
        for col in categorical_columns:
            if col != target_column:
                if col not in self.encoders:
                    self.encoders[col] = LabelEncoder()
                    df[col] = self.encoders[col].fit_transform(df[col].astype(str))
                else:
                    df[col] = self.encoders[col].transform(df[col].astype(str))
        
        return df
    
    def scale_features(self, df, target_column=None):
        """Escalamiento de características"""
        numeric_columns = df.select_dtypes(include=[np.number]).columns
        
        if target_column in numeric_columns:
            numeric_columns = numeric_columns.drop(target_column)
        
        if len(numeric_columns) > 0:
            if 'feature_scaler' not in self.scalers:
                self.scalers['feature_scaler'] = StandardScaler()
                df[numeric_columns] = self.scalers['feature_scaler'].fit_transform(
                    df[numeric_columns]
                )
            else:
                df[numeric_columns] = self.scalers['feature_scaler'].transform(
                    df[numeric_columns]
                )
        
        return df
    
    def split_data(self, df, target_column, test_size=0.2, random_state=42):
        """División de datos en entrenamiento y prueba"""
        X = df.drop(columns=[target_column])
        y = df[target_column]
        
        self.feature_names = list(X.columns)
        
        X_train, X_test, y_train, y_test = train_test_split(
            X, y, test_size=test_size, random_state=random_state, stratify=y
        )
        
        return X_train, X_test, y_train, y_test
    
    def fit_transform(self, df, target_column):
        """Pipeline completo de preprocesamiento"""
        self.logger.info("Iniciando preprocesamiento completo...")
        
        # Limpieza
        df = self.clean_data(df)
        
        # Manejo de valores faltantes
        df = self.handle_missing_values(df)
        
        # Codificación categórica
        df = self.encode_categorical_variables(df, target_column)
        
        # Escalamiento
        df = self.scale_features(df, target_column)
        
        self.logger.info("Preprocesamiento completado")
        return df
    
    def transform(self, df):
        """Transforma nuevos datos usando transformadores ajustados"""
        df = self.handle_missing_values(df)
        df = self.encode_categorical_variables(df)
        df = self.scale_features(df)
        return df
\end{lstlisting}

\section{Módulo de Entrenamiento de Modelos}

\subsection{Entrenador de Modelos}

\begin{lstlisting}[language=Python, caption=Módulo de entrenamiento de modelos]
from sklearn.ensemble import RandomForestClassifier
from sklearn.linear_model import LogisticRegression
from sklearn.svm import SVC
import xgboost as xgb
from sklearn.metrics import classification_report, accuracy_score
import joblib
import json
import time

class ModelTrainer:
    """
    Clase para entrenamiento y evaluación de múltiples modelos
    """
    
    def __init__(self):
        self.models = {}
        self.results = {}
        self.best_model = None
        self.best_score = 0
    
    def initialize_models(self):
        """Inicializa los modelos con configuraciones por defecto"""
        self.models = {
            'logistic_regression': LogisticRegression(random_state=42, max_iter=1000),
            'random_forest': RandomForestClassifier(
                n_estimators=100, random_state=42, n_jobs=-1
            ),
            'svm': SVC(random_state=42, probability=True),
            'xgboost': xgb.XGBClassifier(random_state=42, eval_metric='logloss')
        }
    
    def train_model(self, model_name, X_train, y_train, X_val=None, y_val=None):
        """Entrena un modelo específico"""
        if model_name not in self.models:
            raise ValueError(f"Modelo {model_name} no encontrado")
        
        model = self.models[model_name]
        
        print(f"Entrenando {model_name}...")
        start_time = time.time()
        
        # Entrenar modelo
        if model_name == 'xgboost' and X_val is not None:
            model.fit(
                X_train, y_train,
                eval_set=[(X_val, y_val)],
                early_stopping_rounds=10,
                verbose=False
            )
        else:
            model.fit(X_train, y_train)
        
        training_time = time.time() - start_time
        
        # Evaluar en conjunto de entrenamiento
        train_pred = model.predict(X_train)
        train_accuracy = accuracy_score(y_train, train_pred)
        
        results = {
            'model': model,
            'train_accuracy': train_accuracy,
            'training_time': training_time
        }
        
        # Evaluar en conjunto de validación si está disponible
        if X_val is not None and y_val is not None:
            val_pred = model.predict(X_val)
            val_accuracy = accuracy_score(y_val, val_pred)
            results['val_accuracy'] = val_accuracy
            
            print(f"{model_name} - Train Acc: {train_accuracy:.4f}, Val Acc: {val_accuracy:.4f}")
            
            # Actualizar mejor modelo
            if val_accuracy > self.best_score:
                self.best_score = val_accuracy
                self.best_model = model_name
        else:
            print(f"{model_name} - Train Acc: {train_accuracy:.4f}")
        
        self.results[model_name] = results
        return results
    
    def train_all_models(self, X_train, y_train, X_val=None, y_val=None):
        """Entrena todos los modelos inicializados"""
        self.initialize_models()
        
        for model_name in self.models.keys():
            self.train_model(model_name, X_train, y_train, X_val, y_val)
        
        print(f"\\nMejor modelo: {self.best_model} con accuracy: {self.best_score:.4f}")
        return self.results
    
    def evaluate_model(self, model_name, X_test, y_test):
        """Evalúa un modelo en el conjunto de prueba"""
        if model_name not in self.results:
            raise ValueError(f"Modelo {model_name} no ha sido entrenado")
        
        model = self.results[model_name]['model']
        
        # Predicciones
        y_pred = model.predict(X_test)
        y_proba = model.predict_proba(X_test) if hasattr(model, 'predict_proba') else None
        
        # Métricas
        accuracy = accuracy_score(y_test, y_pred)
        report = classification_report(y_test, y_pred, output_dict=True)
        
        evaluation_results = {
            'accuracy': accuracy,
            'classification_report': report,
            'predictions': y_pred,
            'probabilities': y_proba
        }
        
        print(f"\\n{model_name} - Test Accuracy: {accuracy:.4f}")
        print(classification_report(y_test, y_pred))
        
        return evaluation_results
    
    def save_model(self, model_name, filepath):
        """Guarda un modelo entrenado"""
        if model_name not in self.results:
            raise ValueError(f"Modelo {model_name} no encontrado")
        
        model_data = {
            'model': self.results[model_name]['model'],
            'training_info': {
                'train_accuracy': self.results[model_name]['train_accuracy'],
                'training_time': self.results[model_name]['training_time']
            }
        }
        
        joblib.dump(model_data, filepath)
        print(f"Modelo {model_name} guardado en {filepath}")
    
    def load_model(self, filepath):
        """Carga un modelo previamente guardado"""
        model_data = joblib.load(filepath)
        return model_data['model']
    
    def get_feature_importance(self, model_name, feature_names):
        """Obtiene la importancia de características"""
        if model_name not in self.results:
            raise ValueError(f"Modelo {model_name} no encontrado")
        
        model = self.results[model_name]['model']
        
        if hasattr(model, 'feature_importances_'):
            importance_df = pd.DataFrame({
                'feature': feature_names,
                'importance': model.feature_importances_
            }).sort_values('importance', ascending=False)
            
            return importance_df
        else:
            print(f"Modelo {model_name} no proporciona importancia de características")
            return None
\end{lstlisting}

\section{Módulo de Evaluación}

\subsection{Evaluador de Modelos}

\begin{lstlisting}[language=Python, caption=Módulo de evaluación de modelos]
import matplotlib.pyplot as plt
import seaborn as sns
from sklearn.metrics import confusion_matrix, roc_curve, auc
from sklearn.model_selection import cross_val_score
import pandas as pd
import numpy as np

class ModelEvaluator:
    """
    Clase para evaluación comprehensiva de modelos
    """
    
    def __init__(self, task_type='classification'):
        self.task_type = task_type
        self.evaluation_results = {}
    
    def plot_confusion_matrix(self, y_true, y_pred, model_name, normalize=False):
        """Genera matriz de confusión"""
        cm = confusion_matrix(y_true, y_pred)
        
        if normalize:
            cm = cm.astype('float') / cm.sum(axis=1)[:, np.newaxis]
            fmt = '.2f'
        else:
            fmt = 'd'
        
        plt.figure(figsize=(8, 6))
        sns.heatmap(cm, annot=True, fmt=fmt, cmap='Blues')
        plt.title(f'Matriz de Confusión - {model_name}')
        plt.xlabel('Predicción')
        plt.ylabel('Valor Real')
        plt.show()
        
        return cm
    
    def plot_roc_curve(self, y_true, y_proba, model_name):
        """Genera curva ROC"""
        fpr, tpr, _ = roc_curve(y_true, y_proba[:, 1])
        roc_auc = auc(fpr, tpr)
        
        plt.figure(figsize=(8, 6))
        plt.plot(fpr, tpr, color='darkorange', lw=2, 
                label=f'ROC Curve (AUC = {roc_auc:.2f})')
        plt.plot([0, 1], [0, 1], color='navy', lw=2, linestyle='--')
        plt.xlim([0.0, 1.0])
        plt.ylim([0.0, 1.05])
        plt.xlabel('Tasa de Falsos Positivos')
        plt.ylabel('Tasa de Verdaderos Positivos')
        plt.title(f'Curva ROC - {model_name}')
        plt.legend(loc="lower right")
        plt.show()
        
        return roc_auc
    
    def cross_validate_model(self, model, X, y, cv=5):
        """Validación cruzada"""
        scores = cross_val_score(model, X, y, cv=cv, scoring='accuracy')
        
        results = {
            'scores': scores,
            'mean_score': scores.mean(),
            'std_score': scores.std(),
            'confidence_interval': (
                scores.mean() - 2 * scores.std(),
                scores.mean() + 2 * scores.std()
            )
        }
        
        print(f"Validación Cruzada ({cv}-fold):")
        print(f"Accuracy: {results['mean_score']:.4f} (+/- {results['std_score'] * 2:.4f})")
        print(f"IC 95%: [{results['confidence_interval'][0]:.4f}, {results['confidence_interval'][1]:.4f}]")
        
        return results
    
    def compare_models(self, results_dict):
        """Compara múltiples modelos"""
        comparison_data = []
        
        for model_name, results in results_dict.items():
            comparison_data.append({
                'Modelo': model_name,
                'Accuracy': results.get('accuracy', 0),
                'Precision': results.get('classification_report', {}).get('weighted avg', {}).get('precision', 0),
                'Recall': results.get('classification_report', {}).get('weighted avg', {}).get('recall', 0),
                'F1-Score': results.get('classification_report', {}).get('weighted avg', {}).get('f1-score', 0)
            })
        
        comparison_df = pd.DataFrame(comparison_data)
        comparison_df = comparison_df.sort_values('Accuracy', ascending=False)
        
        # Visualización
        fig, axes = plt.subplots(2, 2, figsize=(15, 10))
        metrics = ['Accuracy', 'Precision', 'Recall', 'F1-Score']
        
        for i, metric in enumerate(metrics):
            row, col = i // 2, i % 2
            axes[row, col].bar(comparison_df['Modelo'], comparison_df[metric])
            axes[row, col].set_title(f'Comparación - {metric}')
            axes[row, col].set_ylabel(metric)
            axes[row, col].tick_params(axis='x', rotation=45)
            
            # Añadir valores sobre las barras
            for j, v in enumerate(comparison_df[metric]):
                axes[row, col].text(j, v + 0.01, f'{v:.3f}', ha='center')
        
        plt.tight_layout()
        plt.show()
        
        return comparison_df
    
    def feature_importance_plot(self, importance_df, top_n=10):
        """Visualiza importancia de características"""
        top_features = importance_df.head(top_n)
        
        plt.figure(figsize=(10, 6))
        sns.barplot(data=top_features, x='importance', y='feature')
        plt.title(f'Top {top_n} Características Más Importantes')
        plt.xlabel('Importancia')
        plt.ylabel('Característica')
        plt.show()
        
        return top_features
    
    def learning_curve_plot(self, model, X, y, train_sizes=None):
        """Genera curvas de aprendizaje"""
        from sklearn.model_selection import learning_curve
        
        if train_sizes is None:
            train_sizes = np.linspace(0.1, 1.0, 10)
        
        train_sizes, train_scores, val_scores = learning_curve(
            model, X, y, train_sizes=train_sizes, cv=5, 
            scoring='accuracy', n_jobs=-1
        )
        
        train_mean = np.mean(train_scores, axis=1)
        train_std = np.std(train_scores, axis=1)
        val_mean = np.mean(val_scores, axis=1)
        val_std = np.std(val_scores, axis=1)
        
        plt.figure(figsize=(10, 6))
        plt.plot(train_sizes, train_mean, 'o-', color='blue', label='Entrenamiento')
        plt.fill_between(train_sizes, train_mean - train_std, train_mean + train_std, 
                        alpha=0.1, color='blue')
        
        plt.plot(train_sizes, val_mean, 'o-', color='red', label='Validación')
        plt.fill_between(train_sizes, val_mean - val_std, val_mean + val_std, 
                        alpha=0.1, color='red')
        
        plt.xlabel('Tamaño del Conjunto de Entrenamiento')
        plt.ylabel('Accuracy')
        plt.title('Curvas de Aprendizaje')
        plt.legend()
        plt.grid(True)
        plt.show()
        
        return train_sizes, train_scores, val_scores
\end{lstlisting}

\section{Script Principal de Ejecución}

\subsection{Pipeline Completo}

\begin{lstlisting}[language=Python, caption=Script principal de ejecución]
#!/usr/bin/env python3
"""
Script principal para ejecutar el pipeline completo de análisis
"""

import pandas as pd
import numpy as np
import logging
from pathlib import Path
import argparse
import yaml

# Imports locales
from src.preprocessing import DataPreprocessor
from src.training import ModelTrainer
from src.evaluation import ModelEvaluator

def setup_logging(log_level='INFO'):
    """Configura el sistema de logging"""
    logging.basicConfig(
        level=getattr(logging, log_level),
        format='%(asctime)s - %(name)s - %(levelname)s - %(message)s',
        handlers=[
            logging.FileHandler('experiment.log'),
            logging.StreamHandler()
        ]
    )

def load_config(config_path):
    """Carga configuración desde archivo YAML"""
    with open(config_path, 'r') as file:
        config = yaml.safe_load(file)
    return config

def main():
    """Función principal del pipeline"""
    parser = argparse.ArgumentParser(description='Pipeline de Analítica de Datos')
    parser.add_argument('--config', type=str, default='config.yaml',
                       help='Archivo de configuración')
    parser.add_argument('--data', type=str, required=True,
                       help='Archivo de datos CSV')
    parser.add_argument('--target', type=str, required=True,
                       help='Columna objetivo')
    parser.add_argument('--output', type=str, default='results/',
                       help='Directorio de salida')
    
    args = parser.parse_args()
    
    # Configurar logging
    setup_logging()
    logger = logging.getLogger(__name__)
    
    try:
        # Cargar configuración
        config = load_config(args.config)
        logger.info(f"Configuración cargada desde {args.config}")
        
        # Crear directorio de salida
        output_dir = Path(args.output)
        output_dir.mkdir(exist_ok=True)
        
        # 1. PREPROCESAMIENTO
        logger.info("=== FASE 1: PREPROCESAMIENTO ===")
        preprocessor = DataPreprocessor(config.get('preprocessing', {}))
        
        # Cargar datos
        df = preprocessor.load_data(args.data)
        logger.info(f"Dataset cargado: {df.shape}")
        
        # Procesar datos
        df_processed = preprocessor.fit_transform(df, args.target)
        
        # Dividir datos
        X_train, X_test, y_train, y_test = preprocessor.split_data(
            df_processed, args.target, 
            test_size=config.get('test_size', 0.2)
        )
        
        # Dividir entrenamiento en train/validation
        from sklearn.model_selection import train_test_split
        X_train, X_val, y_train, y_val = train_test_split(
            X_train, y_train, test_size=0.2, random_state=42
        )
        
        logger.info(f"Datos divididos - Train: {X_train.shape}, Val: {X_val.shape}, Test: {X_test.shape}")
        
        # 2. ENTRENAMIENTO
        logger.info("=== FASE 2: ENTRENAMIENTO ===")
        trainer = ModelTrainer()
        
        # Entrenar todos los modelos
        training_results = trainer.train_all_models(X_train, y_train, X_val, y_val)
        
        # Guardar el mejor modelo
        best_model_path = output_dir / f'best_model_{trainer.best_model}.pkl'
        trainer.save_model(trainer.best_model, best_model_path)
        logger.info(f"Mejor modelo guardado: {best_model_path}")
        
        # 3. EVALUACIÓN
        logger.info("=== FASE 3: EVALUACIÓN ===")
        evaluator = ModelEvaluator()
        
        evaluation_results = {}
        for model_name in training_results.keys():
            eval_result = trainer.evaluate_model(model_name, X_test, y_test)
            evaluation_results[model_name] = eval_result
            
            # Validación cruzada
            cv_results = evaluator.cross_validate_model(
                training_results[model_name]['model'], X_train, y_train
            )
            eval_result['cross_validation'] = cv_results
        
        # Comparar modelos
        logger.info("=== COMPARACIÓN DE MODELOS ===")
        comparison_df = evaluator.compare_models(evaluation_results)
        
        # Guardar resultados
        comparison_path = output_dir / 'model_comparison.csv'
        comparison_df.to_csv(comparison_path, index=False)
        logger.info(f"Comparación guardada: {comparison_path}")
        
        # Análisis de importancia de características
        if trainer.best_model in ['random_forest', 'xgboost']:
            importance_df = trainer.get_feature_importance(
                trainer.best_model, preprocessor.feature_names
            )
            
            if importance_df is not None:
                importance_path = output_dir / 'feature_importance.csv'
                importance_df.to_csv(importance_path, index=False)
                logger.info(f"Importancia de características guardada: {importance_path}")
                
                # Visualizar top características
                evaluator.feature_importance_plot(importance_df)
        
        # 4. RESULTADOS FINALES
        logger.info("=== RESULTADOS FINALES ===")
        best_accuracy = evaluation_results[trainer.best_model]['accuracy']
        logger.info(f"Mejor modelo: {trainer.best_model}")
        logger.info(f"Accuracy en test: {best_accuracy:.4f}")
        
        # Generar reporte final
        report = {
            'best_model': trainer.best_model,
            'test_accuracy': best_accuracy,
            'model_comparison': comparison_df.to_dict('records'),
            'dataset_info': {
                'total_samples': len(df),
                'features': len(preprocessor.feature_names),
                'train_samples': len(X_train),
                'test_samples': len(X_test)
            }
        }
        
        import json
        report_path = output_dir / 'final_report.json'
        with open(report_path, 'w') as f:
            json.dump(report, f, indent=2, default=str)
        
        logger.info(f"Reporte final guardado: {report_path}")
        logger.info("¡Pipeline completado exitosamente!")
        
    except Exception as e:
        logger.error(f"Error en pipeline: {str(e)}")
        raise

if __name__ == "__main__":
    main()
\end{lstlisting}

\section{Archivo de Configuración}

\subsection{Configuración YAML}

\begin{lstlisting}[language=yaml, caption=Archivo de configuración config.yaml]
# Configuración del pipeline de analítica de datos

# Configuración de preprocesamiento
preprocessing:
  missing_value_strategy: "mean"  # mean, median, mode, drop
  outlier_detection: true
  outlier_method: "iqr"  # iqr, zscore, isolation_forest
  scaling_method: "standard"  # standard, minmax, robust
  encoding_method: "label"  # label, onehot, target

# Configuración de división de datos
test_size: 0.2
validation_size: 0.2
random_state: 42
stratify: true

# Configuración de modelos
models:
  logistic_regression:
    enabled: true
    parameters:
      C: [0.1, 1, 10]
      solver: ['liblinear', 'lbfgs']
      max_iter: 1000
  
  random_forest:
    enabled: true
    parameters:
      n_estimators: [100, 200, 500]
      max_depth: [None, 10, 20]
      min_samples_split: [2, 5, 10]
      min_samples_leaf: [1, 2, 4]
  
  svm:
    enabled: true
    parameters:
      C: [0.1, 1, 10, 100]
      kernel: ['linear', 'rbf']
      gamma: ['scale', 'auto']
  
  xgboost:
    enabled: true
    parameters:
      n_estimators: [100, 200, 500]
      max_depth: [3, 6, 9]
      learning_rate: [0.01, 0.1, 0.2]
      subsample: [0.8, 0.9, 1.0]

# Configuración de optimización de hiperparámetros
hyperparameter_optimization:
  method: "random_search"  # grid_search, random_search, bayesian
  n_iter: 50  # Para random_search
  cv_folds: 5
  scoring: "accuracy"
  n_jobs: -1

# Configuración de evaluación
evaluation:
  metrics: ["accuracy", "precision", "recall", "f1", "roc_auc"]
  cross_validation: true
  cv_folds: 5
  confidence_interval: 0.95
  
  # Configuración de visualizaciones
  plots:
    confusion_matrix: true
    roc_curve: true
    feature_importance: true
    learning_curves: true
    model_comparison: true

# Configuración de salida
output:
  save_models: true
  save_preprocessors: true
  save_results: true
  generate_report: true
  plot_format: "png"
  plot_dpi: 300

# Configuración de logging
logging:
  level: "INFO"  # DEBUG, INFO, WARNING, ERROR
  file: "experiment.log"
  format: "%(asctime)s - %(name)s - %(levelname)s - %(message)s"
\end{lstlisting}

\section{Instrucciones de Uso}

\subsection{Instalación de Dependencias}

\begin{lstlisting}[language=bash, caption=Instalación de dependencias]
# Crear entorno virtual
python -m venv venv
source venv/bin/activate  # Linux/Mac
# venv\Scripts\activate  # Windows

# Instalar dependencias
pip install pandas numpy scikit-learn xgboost matplotlib seaborn
pip install jupyter notebook plotly
pip install pyyaml argparse pathlib

# Para desarrollo
pip install pytest black flake8
\end{lstlisting}

\subsection{Ejecución del Pipeline}

\begin{lstlisting}[language=bash, caption=Comandos de ejecución]
# Ejecución básica
python main.py --data datos.csv --target target_column

# Ejecución con configuración personalizada
python main.py --data datos.csv --target target_column --config mi_config.yaml

# Ejecución con directorio de salida específico
python main.py --data datos.csv --target target_column --output resultados/

# Ejecución completa con todos los parámetros
python main.py \
  --data path/to/dataset.csv \
  --target target_variable \
  --config config/experiment.yaml \
  --output results/experiment_1/
\end{lstlisting}

\subsection{Estructura de Archivos}

\begin{lstlisting}[caption=Estructura recomendada del proyecto]
proyecto/
├── src/
│   ├── __init__.py
│   ├── preprocessing.py
│   ├── training.py
│   ├── evaluation.py
│   └── utils.py
├── config/
│   ├── config.yaml
│   └── experiments/
├── data/
│   ├── raw/
│   ├── processed/
│   └── external/
├── models/
│   ├── trained/
│   └── checkpoints/
├── results/
│   ├── figures/
│   ├── reports/
│   └── experiments/
├── notebooks/
│   ├── exploratory/
│   └── analysis/
├── tests/
├── main.py
├── requirements.txt
└── README.md
\end{lstlisting}

% ===============================================
% APÉNDICE B: DATOS Y RESULTADOS COMPLEMENTARIOS
% ===============================================

\chapter{Datos y Resultados Complementarios}
\label{ap:datos_resultados}

% Introducción al apéndice
Este apéndice presenta información complementaria que sustenta los resultados principales de la investigación, incluyendo datasets utilizados, métricas detalladas, configuraciones de modelos y análisis estadísticos adicionales.

% ===============================================
% SECCIÓN B.1: DESCRIPCIÓN DETALLADA DE DATASETS
% ===============================================

\section{Descripción Detallada de Datasets}
\label{sec:datasets_detallados}

\subsection{Dataset Principal}
\label{subsec:dataset_principal}

% Tabla con estadísticas descriptivas del dataset
\begin{table}[htbp]
    \centering
    \caption{Estadísticas Descriptivas del Dataset Principal}
    \label{tab:estadisticas_dataset}
    \begin{tabular}{|l|c|c|c|c|}
        \hline
        \textbf{Variable} & \textbf{Tipo} & \textbf{Valores Únicos} & \textbf{Valores Nulos} & \textbf{Distribución} \\
        \hline
        ID & Numérico & 50,000 & 0 & Secuencial \\
        \hline
        Edad & Numérico & 82 & 150 & Normal \\
        \hline
        Género & Categórico & 3 & 75 & 52\% F, 47\% M, 1\% O \\
        \hline
        Ingresos & Numérico & 45,678 & 892 & Log-normal \\
        \hline
        Región & Categórico & 12 & 23 & Uniforme \\
        \hline
        Target & Binario & 2 & 0 & 35\% Positivo \\
        \hline
    \end{tabular}
\end{table}

\subsection{Datasets de Validación}
\label{subsec:datasets_validacion}

\begin{itemize}
    \item \textbf{Dataset de Test}: 15,000 registros con distribución similar al dataset principal
    \item \textbf{Dataset de Validación Temporal}: 5,000 registros de un período posterior
    \item \textbf{Dataset Benchmark}: Dataset público estándar para comparación (UCI ML Repository)
\end{itemize}

% ===============================================
% SECCIÓN B.2: CONFIGURACIONES DE MODELOS
% ===============================================

\section{Configuraciones Detalladas de Modelos}
\label{sec:configuraciones_modelos}

\subsection{Random Forest}
\label{subsec:config_rf}

\begin{lstlisting}[language=Python, caption=Configuración Random Forest Óptima]
# Configuración encontrada mediante GridSearchCV
rf_config = {
    'n_estimators': 500,
    'max_depth': 15,
    'min_samples_split': 5,
    'min_samples_leaf': 2,
    'max_features': 'sqrt',
    'bootstrap': True,
    'random_state': 42,
    'n_jobs': -1,
    'class_weight': 'balanced'
}
\end{lstlisting}

\subsection{XGBoost}
\label{subsec:config_xgb}

\begin{lstlisting}[language=Python, caption=Configuración XGBoost Óptima]
# Parámetros optimizados con Optuna
xgb_config = {
    'objective': 'binary:logistic',
    'eval_metric': 'auc',
    'max_depth': 8,
    'learning_rate': 0.05,
    'n_estimators': 1000,
    'subsample': 0.8,
    'colsample_bytree': 0.85,
    'reg_alpha': 0.1,
    'reg_lambda': 1.0,
    'scale_pos_weight': 1.85,
    'random_state': 42
}
\end{lstlisting}

\subsection{Red Neuronal}
\label{subsec:config_nn}

\begin{lstlisting}[language=Python, caption=Arquitectura de Red Neuronal]
# Arquitectura de la red neuronal optimizada
model = Sequential([
    Dense(512, activation='relu', input_shape=(n_features,)),
    BatchNormalization(),
    Dropout(0.3),
    
    Dense(256, activation='relu'),
    BatchNormalization(),
    Dropout(0.25),
    
    Dense(128, activation='relu'),
    BatchNormalization(),
    Dropout(0.2),
    
    Dense(64, activation='relu'),
    Dropout(0.15),
    
    Dense(1, activation='sigmoid')
])

# Configuración de entrenamiento
compile_config = {
    'optimizer': Adam(learning_rate=0.001),
    'loss': 'binary_crossentropy',
    'metrics': ['accuracy', 'precision', 'recall']
}
\end{lstlisting}

% ===============================================
% SECCIÓN B.3: RESULTADOS DETALLADOS POR MODELO
% ===============================================

\section{Resultados Detallados por Modelo}
\label{sec:resultados_detallados}

\subsection{Métricas de Evaluación Completas}
\label{subsec:metricas_completas}

% Tabla con todas las métricas
\begin{table}[htbp]
    \centering
    \caption{Métricas Completas de Evaluación por Modelo}
    \label{tab:metricas_completas}
    \resizebox{\textwidth}{!}{%
    \begin{tabular}{|l|c|c|c|c|c|c|c|c|}
        \hline
        \textbf{Modelo} & \textbf{Accuracy} & \textbf{Precision} & \textbf{Recall} & \textbf{F1-Score} & \textbf{AUC-ROC} & \textbf{AUC-PR} & \textbf{Tiempo (s)} & \textbf{Memoria (MB)} \\
        \hline
        Logistic Regression & 0.823 & 0.798 & 0.756 & 0.776 & 0.887 & 0.743 & 2.3 & 45 \\
        \hline
        Random Forest & 0.856 & 0.834 & 0.812 & 0.823 & 0.923 & 0.806 & 45.2 & 512 \\
        \hline
        XGBoost & 0.867 & 0.851 & 0.823 & 0.837 & 0.934 & 0.821 & 67.8 & 256 \\
        \hline
        Red Neuronal & 0.871 & 0.856 & 0.831 & 0.843 & 0.938 & 0.827 & 234.5 & 1024 \\
        \hline
        Ensemble & \textbf{0.879} & \textbf{0.863} & \textbf{0.842} & \textbf{0.852} & \textbf{0.945} & \textbf{0.834} & 89.3 & 892 \\
        \hline
    \end{tabular}
    }
\end{table}

\subsection{Matrices de Confusión}
\label{subsec:matrices_confusion}

% Aquí irían las matrices de confusión para cada modelo
% Ejemplo para el mejor modelo:

\begin{table}[htbp]
    \centering
    \caption{Matriz de Confusión - Modelo Ensemble (Dataset de Test)}
    \label{tab:confusion_ensemble}
    \begin{tabular}{|c|c|c|}
        \hline
        & \textbf{Predicho Negativo} & \textbf{Predicho Positivo} \\
        \hline
        \textbf{Real Negativo} & 8,234 & 516 \\
        \hline
        \textbf{Real Positivo} & 847 & 4,403 \\
        \hline
    \end{tabular}
\end{table}

% ===============================================
% SECCIÓN B.4: ANÁLISIS DE FEATURES
% ===============================================

\section{Análisis Detallado de Features}
\label{sec:analisis_features}

\subsection{Importancia de Variables}
\label{subsec:importancia_variables}

% Tabla con importancia de features
\begin{table}[htbp]
    \centering
    \caption{Top 15 Features más Importantes (Random Forest)}
    \label{tab:feature_importance}
    \begin{tabular}{|l|c|c|}
        \hline
        \textbf{Feature} & \textbf{Importancia} & \textbf{Importancia Normalizada} \\
        \hline
        historial\_transacciones & 0.156 & 1.000 \\
        \hline
        edad\_cliente & 0.134 & 0.859 \\
        \hline
        ingresos\_anuales & 0.098 & 0.628 \\
        \hline
        score\_crediticio & 0.087 & 0.558 \\
        \hline
        tiempo\_cliente & 0.076 & 0.487 \\
        \hline
        productos\_activos & 0.065 & 0.417 \\
        \hline
        frecuencia\_uso & 0.058 & 0.372 \\
        \hline
        ubicacion\_geografica & 0.045 & 0.288 \\
        \hline
        canal\_preferido & 0.039 & 0.250 \\
        \hline
        satisfaccion\_cliente & 0.034 & 0.218 \\
        \hline
        interacciones\_soporte & 0.031 & 0.199 \\
        \hline
        dispositivo\_principal & 0.028 & 0.179 \\
        \hline
        promociones\_usadas & 0.025 & 0.160 \\
        \hline
        referencias\_sociales & 0.022 & 0.141 \\
        \hline
        estado\_civil & 0.019 & 0.122 \\
        \hline
    \end{tabular}
\end{table}

\subsection{Análisis de Correlaciones}
\label{subsec:correlaciones}

% Descripción de correlaciones importantes encontradas
Las correlaciones más significativas encontradas en el análisis exploratorio:

\begin{itemize}
    \item \textbf{historial\_transacciones} vs \textbf{target}: r = 0.68 (correlación fuerte positiva)
    \item \textbf{edad\_cliente} vs \textbf{ingresos\_anuales}: r = 0.45 (correlación moderada positiva)
    \item \textbf{score\_crediticio} vs \textbf{productos\_activos}: r = 0.52 (correlación moderada positiva)
    \item \textbf{frecuencia\_uso} vs \textbf{satisfaccion\_cliente}: r = 0.39 (correlación moderada positiva)
\end{itemize}

% ===============================================
% SECCIÓN B.5: VALIDACIÓN CRUZADA DETALLADA
% ===============================================

\section{Validación Cruzada Detallada}
\label{sec:validacion_cruzada}

\subsection{Resultados por Fold}
\label{subsec:resultados_fold}

% Tabla con resultados de validación cruzada
\begin{table}[htbp]
    \centering
    \caption{Resultados de Validación Cruzada (K=5) - Modelo Final}
    \label{tab:cv_results}
    \begin{tabular}{|c|c|c|c|c|}
        \hline
        \textbf{Fold} & \textbf{Accuracy} & \textbf{Precision} & \textbf{Recall} & \textbf{F1-Score} \\
        \hline
        1 & 0.874 & 0.859 & 0.838 & 0.848 \\
        \hline
        2 & 0.881 & 0.867 & 0.845 & 0.856 \\
        \hline
        3 & 0.876 & 0.862 & 0.841 & 0.851 \\
        \hline
        4 & 0.883 & 0.868 & 0.847 & 0.857 \\
        \hline
        5 & 0.878 & 0.864 & 0.843 & 0.853 \\
        \hline
        \textbf{Media} & \textbf{0.878} & \textbf{0.864} & \textbf{0.843} & \textbf{0.853} \\
        \hline
        \textbf{Std} & \textbf{0.003} & \textbf{0.004} & \textbf{0.003} & \textbf{0.004} \\
        \hline
    \end{tabular}
\end{table}

% ===============================================
% SECCIÓN B.6: ANÁLISIS DE ERRORES
% ===============================================

\section{Análisis de Errores y Casos Límite}
\label{sec:analisis_errores}

\subsection{Caracterización de Falsos Positivos}
\label{subsec:falsos_positivos}

Análisis de los casos donde el modelo predice incorrectamente la clase positiva:

\begin{itemize}
    \item \textbf{Patrón 1}: Clientes jóvenes con historial limitado (23\% de FP)
    \item \textbf{Patrón 2}: Cambios recientes en comportamiento (19\% de FP)
    \item \textbf{Patrón 3}: Datos inconsistentes o ruidosos (15\% de FP)
    \item \textbf{Otros patrones}: (43\% de FP)
\end{itemize}

\subsection{Caracterización de Falsos Negativos}
\label{subsec:falsos_negativos}

Análisis de los casos donde el modelo no detecta correctamente la clase positiva:

\begin{itemize}
    \item \textbf{Patrón 1}: Comportamiento atípico no capturado por features (31\% de FN)
    \item \textbf{Patrón 2}: Clientes en transición de segmento (24\% de FN)
    \item \textbf{Patrón 3}: Eventos externos no modelados (18\% de FN)
    \item \textbf{Otros patrones}: (27\% de FN)
\end{itemize}

% ===============================================
% SECCIÓN B.7: CONFIGURACIÓN DEL ENTORNO
% ===============================================

\section{Configuración del Entorno de Desarrollo}
\label{sec:entorno_desarrollo}

\subsection{Especificaciones del Hardware}
\label{subsec:hardware}

\begin{itemize}
    \item \textbf{Procesador}: Intel i7-10700K @ 3.80GHz (8 cores, 16 threads)
    \item \textbf{Memoria RAM}: 32 GB DDR4 3200MHz
    \item \textbf{GPU}: NVIDIA RTX 3080 (10GB VRAM) - para modelos de deep learning
    \item \textbf{Almacenamiento}: SSD NVMe 1TB para datos y modelos
    \item \textbf{Sistema Operativo}: Ubuntu 20.04 LTS
\end{itemize}

\subsection{Versiones de Software}
\label{subsec:software}

\begin{lstlisting}[language=bash, caption=Versiones de Paquetes Principales]
# Python y librerías principales
Python: 3.9.7
NumPy: 1.21.2
Pandas: 1.3.3
Scikit-learn: 1.0.2
XGBoost: 1.5.1
TensorFlow: 2.7.0
Keras: 2.7.0

# Visualización y análisis
Matplotlib: 3.4.3
Seaborn: 0.11.2
Plotly: 5.3.1

# Utilidades
Jupyter: 1.0.0
optuna: 2.10.0
mlflow: 1.20.2
\end{lstlisting}

% ===============================================
% SECCIÓN B.8: INSTRUCCIONES DE REPRODUCIBILIDAD
% ===============================================

\section{Instrucciones de Reproducibilidad}
\label{sec:reproducibilidad}

\subsection{Pasos para Reproducir los Resultados}
\label{subsec:pasos_reproduccion}

\begin{enumerate}
    \item \textbf{Preparación del entorno}:
    \begin{lstlisting}[language=bash]
# Crear entorno virtual
python -m venv tesis_env
source tesis_env/bin/activate  # Linux/Mac
# tesis_env\Scripts\activate   # Windows

# Instalar dependencias
pip install -r requirements.txt
    \end{lstlisting}
    
    \item \textbf{Descarga y preparación de datos}:
    \begin{lstlisting}[language=bash]
# Ejecutar script de descarga
python scripts/download_data.py

# Preprocesar datos
python scripts/preprocess_data.py --config config/preprocess.yaml
    \end{lstlisting}
    
    \item \textbf{Entrenamiento de modelos}:
    \begin{lstlisting}[language=bash]
# Entrenar todos los modelos
python scripts/train_models.py --experiment final_experiment

# O entrenar modelo específico
python scripts/train_single_model.py --model xgboost --config config/xgb_final.yaml
    \end{lstlisting}
    
    \item \textbf{Evaluación y resultados}:
    \begin{lstlisting}[language=bash]
# Generar evaluaciones
python scripts/evaluate_models.py --experiment final_experiment

# Generar reportes
python scripts/generate_reports.py --output reports/
    \end{lstlisting}
\end{enumerate}

\subsection{Semillas Aleatorias}
\label{subsec:semillas}

Para garantizar la reproducibilidad completa, se utilizaron las siguientes semillas:

\begin{lstlisting}[language=Python, caption=Configuración de Semillas Aleatorias]
import random
import numpy as np
import tensorflow as tf
from sklearn.utils import check_random_state

# Semilla principal
RANDOM_SEED = 42

# Configurar todas las fuentes de aleatoriedad
random.seed(RANDOM_SEED)
np.random.seed(RANDOM_SEED)
tf.random.set_seed(RANDOM_SEED)

# Para scikit-learn
random_state = check_random_state(RANDOM_SEED)
\end{lstlisting}

Este apéndice proporciona toda la información necesaria para comprender, validar y reproducir los resultados obtenidos en esta investigación.


% Índice alfabético (opcional)
\printindex

\end{document}
